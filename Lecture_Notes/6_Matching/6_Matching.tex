\documentclass[12pt]{article}

% --- Paquetes ---
\usepackage{pifont} 
\usepackage{tikz}
\usetikzlibrary{arrows.meta}
\usepackage{pgfplots}
\pgfplotsset{compat=1.18}
\usepackage[most]{tcolorbox}
\usepackage[spanish,es-tabla]{babel}   % español
\usepackage[utf8]{inputenc}            % acentos
\usepackage[T1]{fontenc}
\usepackage{lmodern}
\usepackage{geometry}
\usepackage{fancyhdr}
\usepackage{xcolor}
\usepackage{titlesec}
\usepackage{lastpage}
\usepackage{amsmath,amssymb}
\usepackage{enumitem}
\usepackage[table]{xcolor} % para \cellcolor y \rowcolor
\usepackage{colortbl}      % colores en tablas
\usepackage{float}         % para usar [H] si quieres fijar la tabla
\usepackage{array}         % mejor control de columnas
\usepackage{amssymb}       % para palomita
\usepackage{graphicx}      % para logo github
\usepackage{hyperref}
\usepackage{setspace} % para hipervinculo
\usepackage[normalem]{ulem}
\usepackage{siunitx}       % Asegúrate de tener este paquete en el preámbulo
\usepackage{booktabs}
\sisetup{
    output-decimal-marker = {.},
    group-separator = {,},
    group-minimum-digits = 4,
    detect-all
}

% Etiqueta en el caption (en la tabla misma)
\usepackage{caption}
\captionsetup[table]{name=Tabla, labelfont=bf, labelsep=period}

% Prefijo en la *Lista de tablas*
\usepackage{tocloft}
\renewcommand{\cfttabpresnum}{Tabla~} % texto antes del número
\renewcommand{\cfttabaftersnum}{.}    % punto después del número
\setlength{\cfttabnumwidth}{5em}      % ancho para "Tabla 10." ajusta si hace falta



% --- Márgenes y encabezado ---
\geometry{left=1in, right=1in, top=1in, bottom=1in}

% Alturas del encabezado (un poco más por las 2–3 líneas del header)
\setlength{\headheight}{32pt}
\setlength{\headsep}{20pt}

\definecolor{maroon}{RGB}{128, 0, 0}

\pagestyle{fancy}
\fancyhf{}

% Regla del encabezado (opcional)
\renewcommand{\headrulewidth}{0.4pt}

% Encabezado izquierdo
\fancyhead[L]{%
  \textcolor{maroon}{\textbf{El Colegio de México}}\\
  \textbf{Microeconometrics for Evaluation}
}

% Encabezado derecho
\fancyhead[R]{%
  6 Matching\\
  \textbf{Jose Daniel Fuentes García}\\
  Github : \includegraphics[height=1em]{github.png}~\href{https://github.com/danifuentesga}{\texttt{danifuentesga}}
}

% Número de página al centro del pie
\fancyfoot[C]{\thepage}

% --- APLICAR EL MISMO ESTILO A PÁGINAS "PLAIN" (TOC, LOT, LOF) ---
\fancypagestyle{plain}{%
  \fancyhf{}
  \renewcommand{\headrulewidth}{0.4pt}
  \fancyhead[L]{%
    \textcolor{maroon}{\textbf{El Colegio de México}}\\
    \textbf{Microeconometrics for Evaluation}
  }
  \fancyhead[R]{%
   6 Matching\\
    \textbf{Jose Daniel Fuentes García}\\
    Github : \includegraphics[height=1em]{github.png}~\href{https://github.com/danifuentesga}{\texttt{danifuentesga}}
  }
  \fancyfoot[C]{\thepage}
}

% Pie de página centrado
\fancyfoot[C]{\thepage\ de \pageref{LastPage}}

\renewcommand{\headrulewidth}{0.4pt}

% --- Color principal ---
\definecolor{formalblue}{RGB}{0,51,102} % azul marino sobrio

% --- Estilo de títulos ---
\titleformat{\section}[hang]{\bfseries\Large\color{formalblue}}{}{0em}{}[\titlerule]
\titleformat{\subsection}{\bfseries\large\color{formalblue}}{\thesubsection}{1em}{}


% --- Listas ---
\setlist[itemize]{leftmargin=1.2em}

% --- Sin portada ---
\title{}
\author{}
\date{}

\begin{document}

\begin{titlepage}
    \vspace*{-1cm}
    \noindent
    \begin{minipage}[t]{0.49\textwidth}
        \includegraphics[height=2.2cm]{colmex.jpg}
    \end{minipage}%
    \begin{minipage}[t]{0.49\textwidth}
        \raggedleft
        \includegraphics[height=2.2cm]{cee.jpg}
    \end{minipage}

    \vspace*{2cm}

    \begin{center}
        \Huge \textbf{CENTRO DE ESTUDIOS ECONÓMICOS} \\[1.5em]
        \Large Maestría en Economía 2024--2026 \\[2em]
        \Large Microeconometrics for Evaluation \\[3em]
        \LARGE \textbf{6 Matching} \\[3em]
        \large \textbf{Disclaimer:} I AM NOT the original intellectual author of the material presented in these notes. The content is STRONGLY based on a combination of lecture notes (from Aurora Ramirez), textbook references, and personal annotations for learning purposes. Any errors or omissions are entirely my own responsibility.\\[0.9em]
        
    \end{center}

    \vfill
\end{titlepage}

\newpage

\setcounter{secnumdepth}{2}
\setcounter{tocdepth}{4}
\tableofcontents

\newpage

\section*{\noindent\textbf{Today’s Agenda}}
\addcontentsline{toc}{section}{Today’s Agenda}

\begin{itemize}
    \item \textbf{Matching}. We search for \textit{causal effects} by comparing \textbf{treated} and \textbf{control groups} inside subgroups where almost everything \ldots or most factors \ldots or the \textit{key elements} \ldots stay fixed
    \item \textbf{Similarities} between \textit{regression} approaches and \textbf{matching}
    \item \textbf{Example} of matching from \textit{Angrist (1998)}
\end{itemize}

\textbf{Intuition}
\begin{itemize}
    \item Matching is a way to \textit{imitate experiments} by checking “similar” groups
    \item It works by \textbf{holding constant} the most \textit{relevant characteristics}
    \item Today’s focus: learn the method and see a \textbf{practical example}
\end{itemize}

\section*{\noindent\textbf{Background: Volunteers of America!}}
\addcontentsline{toc}{section}{Background: Volunteers of America!}

\begin{itemize}
    \item The \textbf{army} is the largest single employer of \textit{young men and women} in the United States
    \item Between \textbf{1989 and 1992}, the enlistments of men and women with \textit{no prior service} in the military dropped by \textbf{27\%}
    \item Recruitments of \textbf{white men} declined by \textbf{25\%}, while enlistments of \textbf{black men}—the group most affected by the \textit{military cutback}—fell by \textbf{47\%} (\textit{Angrist, 1993a})
    \item The main channel behind these decreases was a \textbf{raise in cutoff scores} on applicant tests and \textit{changes in entry rules}
\end{itemize}

\textbf{Intuition}
\begin{itemize}
    \item The military is a \textbf{big employer}, so changes hit many people
    \item During cuts, enlistments fell, especially among \textit{black recruits}
    \item Rules got tougher: \textbf{higher test thresholds} and stricter entry norms
\end{itemize}

\begin{itemize}
    \item What were the \textbf{consequences} of military service for the \textbf{recruits}?
    \item By answering this, we learn whether the \textbf{military cuts} were a \textbf{lost economic opportunity} (as many believed at that time)
    \item The problem of \textbf{selection bias} makes comparisons between \textbf{veterans} and \textbf{non-veterans} misleading (\textbf{Seltzer and Jablon, 1974})
\end{itemize}

\textbf{Intuition}
\begin{itemize}
    \item The key question: did service \textbf{help or hurt} recruits?
    \item Cuts may mean \textbf{missed chances} for economic gains
    \item We must beware of \textbf{biased comparisons} between groups
\end{itemize}

\section*{\noindent\textbf{Angrist (1998). Matching Strategy}}
\addcontentsline{toc}{section}{Angrist (1998). Matching Strategy}

\begin{enumerate}
    \item Compare \textbf{veterans} and \textbf{non-veterans} who applied (only half of the \textbf{qualified candidates} serve in the army)
    \item Control for the \textbf{characteristics} that the military uses to \textbf{select} soldiers
\end{enumerate}

\begin{itemize}
    \item The \textbf{matching estimator} is an average of \textbf{contrasts} or comparisons across cells defined by \textbf{covariates}
    \item Focusing on the \textbf{treatment effect} on the treated (TOT):  
    \[
    E[Y_{1i} - Y_{0i} \mid D_i = 1]
    \]
    \item This shows the gap between the \textbf{average observed earnings} of soldiers, \(E[Y_{1i}\mid D_i=1]\), and the \textbf{counterfactual average} had they not served, \(E[Y_{0i}\mid D_i=1]\)
\end{itemize}

\textbf{Intuition}
\begin{itemize}
    \item We compare \textbf{similar applicants} to isolate service effects
    \item Matching builds a \textbf{fair comparison} by holding traits constant
    \item The goal is to measure the \textbf{earnings gap} caused by service
\end{itemize}

\begin{itemize}
    \item The \textbf{income gap} by veteran status is a \textbf{biased estimator} of the TOT, unless \(D_i\) is independent of \(Y_{0i}\):
\end{itemize}

\singlespacing
\begin{align}
\mathbb{E}[Y_i \mid D_i=1] - \mathbb{E}[Y_i \mid D_i=0] 
   &\quad \text{\textbf{\shortstack{Diff. \\ means}}} \\[6pt]
= \mathbb{E}[Y_{1i} \mid D_i=1] - \mathbb{E}[Y_{0i} \mid D_i=0] 
   &\quad \text{\textbf{\shortstack{Def. of \\ $Y_i$}}} \\[6pt]
= \big(\mathbb{E}[Y_{1i} \mid D_i=1] - \mathbb{E}[Y_{0i} \mid D_i=1]\big) 
  + \big(\mathbb{E}[Y_{0i} \mid D_i=1] - \mathbb{E}[Y_{0i} \mid D_i=0]\big) 
   &\quad \text{\textbf{\shortstack{Add \& subtract \\ $\mathbb{E}[Y_{0i}\mid D_i=1]$}}} \\[6pt]
= \mathbb{E}[Y_{1i} - Y_{0i} \mid D_i=1] 
  + \big(\mathbb{E}[Y_{0i} \mid D_i=1] - \mathbb{E}[Y_{0i} \mid D_i=0]\big) 
   &\quad \text{\textbf{\shortstack{Group \\ terms}}} \\[6pt]
= \underbrace{\mathbb{E}[Y_{1i} - Y_{0i} \mid D_i=1]}_{\textbf{TOT}} 
  + \underbrace{\big(\mathbb{E}[Y_{0i} \mid D_i=1] - \mathbb{E}[Y_{0i} \mid D_i=0]\big)}_{\textbf{Selection Bias}} 
   &\quad \text{\textbf{\shortstack{Identify \\ components}}}
\end{align}


\textbf{Intuition}
\begin{itemize}
    \item The raw gap mixes \textbf{treatment effect} and \textbf{bias}
    \item \textbf{TOT}: what veterans gain from service
    \item \textbf{Bias}: pre-existing differences in \(\;Y_0\) across groups
\end{itemize}

\section*{\noindent\textbf{Conditional Independence Assumption (CIA)}}
\addcontentsline{toc}{section}{Conditional Independence Assumption (CIA)}

\begin{itemize}
    \item Conditional on observed \(\;X_i\), the treatment is as good as randomly assigned:  
    \[
    \{Y_{0i}, Y_{1i}\} \perp D_i \mid X_i
    \]
    \item Given CIA, causal effects can be built by iterating expectations over \(X_i\):
\end{itemize}

\singlespacing
\begin{align}
\delta_{TOT} 
   &= \mathbb{E}[Y_{1i} - Y_{0i} \mid D_i = 1] 
   & \text{\textbf{\shortstack{Def. \\ TOT}}} \\[6pt]
&= \mathbb{E}[Y_{1i} \mid D_i=1] - \mathbb{E}[Y_{0i} \mid D_i=1] 
   & \text{\textbf{\shortstack{Linearity \\ of $\mathbb{E}$}}} \\[6pt]
&= \mathbb{E}\big[ \mathbb{E}[Y_{1i}\mid X_i, D_i=1] \mid D_i=1 \big] 
 - \mathbb{E}\big[ \mathbb{E}[Y_{0i}\mid X_i, D_i=1] \mid D_i=1 \big] 
   & \text{\textbf{\shortstack{Iterated \\ Exp.}}} \\[6pt]
&= \mathbb{E}\Big[ \, 
   \mathbb{E}[Y_{1i}\mid X_i, D_i=1] - \mathbb{E}[Y_{0i}\mid X_i, D_i=1] 
   \; \Big| \; D_i=1 \Big] 
   & \text{\textbf{\shortstack{Group \\ terms}}} \\[6pt]
&= \int \Big( \mathbb{E}[Y_{1i}\mid X_i, D_i=1] - \mathbb{E}[Y_{0i}\mid X_i, D_i=1] \Big) 
   dF(X_i \mid D_i=1) 
   & \quad \text{\textbf{\shortstack{Integral form: \\ average over \\ distribution of $X_i$}}}
\end{align}

\begin{align}
\mathbb{E}[Y_{0i}\mid X_i, D_i=1] 
   & \quad \text{\textbf{Counterfactual term}} \\[6pt]
&\overset{CIA}{=} \mathbb{E}[Y_{0i}\mid X_i, D_i=0] 
   & \text{\textbf{\shortstack{Use \\ CIA}}}
\end{align}

\begin{align}
\delta_{TOT} 
   &= \mathbb{E}\Big[ \mathbb{E}[Y_{1i}\mid X_i, D_i=1] - \mathbb{E}[Y_{0i}\mid X_i, D_i=0] 
      \;\Big|\; D_i=1 \Big] 
   & \text{\textbf{\shortstack{Replace \\ $Y_0$ term}}} \\[6pt]
&= \mathbb{E}[\delta_{X_i} \mid D_i=1] 
   & \text{\textbf{\shortstack{Define \\ $\delta_{X_i}$}}}
\end{align}


\begin{itemize}
    \item where \(\delta_{X_i} = \mathbb{E}[Y_{1i}\mid X_i, D_i=1] - \mathbb{E}[Y_{0i}\mid X_i, D_i=0]\)  
    is the (random) \textbf{X-specific causal effect}.
\end{itemize}

\textbf{Intuition}
\begin{itemize}
    \item CIA says: once we \textbf{control for \(X_i\)}, treatment is \textbf{“as if random”}
    \item TOT is built by \textbf{averaging treatment effects} across \(X_i\)
    \item Counterfactuals are replaced by \textbf{non-treated with same \(X_i\)}
\end{itemize}

\section*{\noindent\textbf{Angrist (1998)}}
\addcontentsline{toc}{section}{Angrist (1998)}

\begin{itemize}
    \item Angrist (1998) builds the \textbf{sample analog} of the right-hand side of eq. (1) for \textbf{discrete covariates}:
\end{itemize}

\singlespacing
\begin{align}
\mathbb{E}[Y_{1i} - Y_{0i} \mid D_i = 1] 
   &= \sum_x \delta_x \, P(X_i = x \mid D_i=1) 
   & \text{\textbf{\shortstack{Discrete \\ version}}} \\[6pt]
\delta_x 
   &= \mathbb{E}[Y_{1i} \mid X_i = x, D_i=1] - \mathbb{E}[Y_{0i} \mid X_i = x, D_i=0] 
   & \text{\textbf{\shortstack{Cell-specific \\ effect}}}
\end{align}

\begin{itemize}
    \item Here \(P(X_i = x \mid D_i = 1)\) is the \textbf{distribution of \(X_i\)} among veterans
    \item The estimator is obtained by replacing \(\delta_x\) with the \textbf{mean earnings difference} between veterans and non-veterans in each cell \(x\), and weighting by the \textbf{empirical distribution} \(P(X_i = x \mid D_i=1)\)
\end{itemize}

\textbf{Intuition}
\begin{itemize}
    \item With discrete \(X\), the integral turns into a \textbf{sum over cells}
    \item Each \(\delta_x\) captures the \textbf{local treatment effect} for covariate profile \(x\)
    \item The final TOT is a \textbf{weighted average} of these cell effects
\end{itemize}

\section*{\noindent\textbf{Angrist (1998). Results}}
\addcontentsline{toc}{section}{Angrist (1998). Results}

\begin{itemize}
    \item \textbf{White veterans} earn more than non-veterans, but this effect becomes \textbf{negative} once covariates are matched
    \item \textbf{Non-white veterans} earn much more than non-veterans, but covariate controls \textbf{shrink the gap} considerably
\end{itemize}

\textbf{Intuition}
\begin{itemize}
    \item At first glance, veterans seem to earn \textbf{more}, but once we compare \textbf{similar people}, the advantage often \textbf{disappears or reverses}
    \item Think of it like comparing two runners: one starts with a head start. After correcting for that, the real performance looks different
    \item For non-white veterans, the raw difference looks huge, but careful matching shows it’s \textbf{partly explained by background factors}
\end{itemize}

\begin{table}[H]
\centering
\caption{Applicant Population and Sample}
\begin{tabular}{lccccccc}
\hline
 & 1976 & 1977 & 1978 & 1979 & 1980 & 1981 & 1982 \\
\hline
\multicolumn{8}{l}{\textbf{A. Population$^a$}} \\
White & 393.5 & 286.9 & 235.9 & 253.1 & 348.6 & 387.3 & 309.8 \\
Percent veteran$^b$ & 53 & 52 & 54 & 55 & 53 & 49 & 52 \\
Nonwhite & 128.6 & 114.8 & 103.6 & 119.5 & 134.3 & 149.3 & 112.5 \\
Percent veteran & 44 & 46 & 50 & 46 & 41 & 36 & 43 \\
\hline
\multicolumn{8}{l}{\textbf{B. Sample$^c$}} \\
White & 49.2 & 46.5 & 40.0 & 39.4 & 52.9 & 57.9 & 47.3 \\
Percent veteran & 56 & 53 & 55 & 57 & 54 & 50 & 53 \\
Nonwhite & 50.9 & 48.1 & 44.6 & 51.9 & 57.0 & 63.7 & 48.7 \\
Percent veteran & 40 & 38 & 44 & 48 & 47 & 38 & 45 \\
\hline
\end{tabular}

\begin{flushleft}
\footnotesize 
$^a$ As in Angrist (1993a, Table 4), excluding applicants with less than 9th grade education at application. Numbers in thousands. \\
$^b$ Veterans are applicants identified as entrants to the military within two years following application. \\
$^c$ Approximately 90\% of the sample is self-weighting, conditional on race. \\
\end{flushleft}
\end{table}

\doublespacing
\textbf{Explanation} \\
Table 1 reports the size of the applicant pool and the sample used in the analysis, 
split by \textbf{race} and \textbf{year of application} (1976--1982). 
Panel A shows the total number of applicants (in thousands) and the share who are 
\textbf{veterans}. Panel B shows the selected sample of applicants who actually 
enlisted, again by race and year, with corresponding veteran shares. 
\singlespacing
\textbf{Intuition}
\begin{itemize}
    \item White applicants are the majority, but the share of \textbf{veterans} is similar across races
    \item The \textbf{sample} is much smaller than the population, since not all applicants enlist
    \item Think of it like a \textbf{funnel}: many apply (Panel A), fewer serve (Panel B). 
    This shows why careful sampling matters in the analysis
\end{itemize}

\begin{figure}[H]
\centering
\includegraphics[width=0.85\textwidth]{image1}
\caption*{\footnotesize 
Earnings profiles by veteran status and application year for men who applied 1979--82, 
with AFQT scores in categories III and IV. The plot shows actual earnings: 
+ \$3{,}000 for 1982 applicants, + \$6{,}000 for 1981 applicants, 
+ \$6{,}000 for 1980 applicants, and + \$9{,}000 for 1979 applicants.}
\end{figure}

\doublespacing
\textbf{Explanation} \\
The figure plots \textbf{earnings profiles} over time for veterans and nonveterans, 
by year of application (1979--1982). Each curve shows the actual FICA earnings for men 
with AFQT scores in categories III and IV. Veterans consistently earn more than 
nonveterans in the early years, with the earnings advantage varying by application cohort.

\singlespacing
\textbf{Intuition}
\begin{itemize}
    \item The \textbf{gap} between the solid (veterans) and dashed (nonveterans) lines 
    shows how much extra veterans earned
    \item Earlier cohorts (like 1979) had the \textbf{largest boost}, while later ones (1982) saw smaller gains
    \item Think of it like different \textbf{graduating classes}: those who entered earlier 
    cashed in more, while later classes faced lower returns
\end{itemize}

\begin{table}[H]
\centering
\caption{Alternative Estimates of the Effects of Military Service}
\scriptsize
\begin{tabular}{lcccccccc}
\hline
 & \multicolumn{4}{c}{\textbf{Whites}} & \multicolumn{4}{c}{\textbf{Nonwhites}} \\
\cline{2-5} \cline{6-9}
Year & Mean 
     & \shortstack{Diff. \\ in Means} 
     & \shortstack{Controlled \\ Contrast} 
     & \shortstack{Regression \\ Est.} 
     & Mean 
     & \shortstack{Diff. \\ in Means} 
     & \shortstack{Controlled \\ Contrast} 
     & \shortstack{Regression \\ Est.} \\
\hline
74 & 182.7  & -26.1 & -14.0 & -13.0 & 157.2  & -4.9 & -2.0 & -3.9 \\
75 & 237.9  & -41.4 & -14.2 & -12.0 & 216.9  & -6.6 & -17.1 & -15.2 \\
76 & 473.4  & -47.9 & -14.8 & -12.7 & 413.6  & -14.5 & -33.3 & -30.2 \\
77 & 1012.9 & -7.1  & -8.6  & -9.4  & 820.9  & -13.0 & -56.0 & -51.3 \\
78 & 2147.1 & 40.3  & -23.5 & -22.4 & 1677.9 & 58.1 & -53.6 & -42.5 \\
79 & 3560.7 & 188.0 & -8.4  & -11.2 & 2797.0 & 340.3 & 119.1 & 122.3 \\
80 & 4790.0 & 572.9 & 178.0 & 175.9 & 3932.2 & 1154.3 & 741.6 & 738.5 \\
81 & 6226.0 & 855.5 & 249.5 & 249.9 & 5218.8 & 1920.0 & 1299.9 & 1318.5 \\
82 & 7200.6 & 1508.5 & 783.3 & 782.4 & 6150.2 & 2917.1 & 2186.0 & 2210.1 \\
83 & 7957.3 & 1305.3 & 584.4 & 532.5 & 6852.1 & 2834.9 & 2320.0 & 2260.0 \\
84 & 9874.2 & 1593.5 & 888.6 & 805.1 & 8377.2 & 2902.7 & 1330.6 & 1289.2 \\
85 & 10972.7 & 2097.2 & 1281.3 & 1258.9 & 9306.8 & 3555.5 & 1942.3 & 1939.2 \\
86 & 12004.5 & 543.7 & -557.3 & -491.7 & 10106.2 & 1381.3 & 720.9 & 872.3 \\
87 & 13045.7 & 654.3 & -598.0 & -464.3 & 10833.0 & 2050.1 & 751.0 & 925.4 \\
88 & 14006.4 & 593.0 & -440.8 & -333.5 & 11506.7 & 2158.2 & 858.0 & 924.3 \\
89 & 14117.5 & -61.0 & -71.2 & -62.9 & 11751.4 & 843.2 & 189.6 & 267.9 \\
90 & 14886.1 & -139.8 & -166.2 & -139.6 & 11904.3 & 2483.6 & 624.9 & 1064.0 \\
91 & 14407.9 & 1559.6 & 29.8 & 16.2 & 11518.7 & 2758.8 & 1062.1 & 1277.9 \\
\hline
\end{tabular}
\end{table}

\doublespacing
\textbf{Explanation} \\
Table 2 reports alternative estimates of the effect of \textbf{military service} on 
earnings, shown separately for \textbf{whites} and \textbf{nonwhites}, by year of application.  
Each panel shows four measures: the \textbf{mean earnings}, the simple 
\textbf{difference in means} between veterans and nonveterans, a 
\textbf{controlled contrast} adjusting for covariates, and a 
\textbf{regression estimate}. Standard errors are reported in parentheses in the original source.

\singlespacing
\textbf{Intuition}
\begin{itemize}
    \item Raw comparisons (diff. in means) often exaggerate or mislead; once we 
    \textbf{control for covariates}, the effects change noticeably
    \item For whites, the estimated benefit is sometimes small or even 
    \textbf{negative}, while for nonwhites the veteran premium is usually 
    \textbf{larger and more positive}
    \item Think of it like comparing test scores: if we just compare averages, 
    one group looks much better — but once we compare students with the 
    \textbf{same background}, the real difference is smaller (and sometimes flips sign)
\end{itemize}

\begin{figure}[H]
\centering
\includegraphics[width=0.85\textwidth]{image2}
\caption*{\footnotesize 
Controlled contrasts by application year and calendar year for 
(a) whites and (b) nonwhites.}
\end{figure}

\doublespacing
\textbf{Explanation} \\
Figure 3 shows the \textbf{controlled contrasts} in earnings by year of application and 
calendar year, separately for \textbf{whites} (panel a) and \textbf{nonwhites} (panel b).  
Each line corresponds to a different application cohort (1979--1982).  
The vertical axis measures the earnings gap between veterans and nonveterans after 
controlling for covariates.  

\singlespacing
\textbf{Intuition}
\begin{itemize}
    \item The veteran \textbf{premium} rises sharply in the early 1980s and then 
    gradually declines for both groups
    \item For whites, the boost is smaller and sometimes disappears; for nonwhites, 
    the peak is \textbf{larger and more sustained}
    \item Think of it like \textbf{waves of opportunity}: earlier cohorts (1979--1980) 
    rode a higher wave, while later ones (1982) caught a smaller one
\end{itemize}

\begin{table}[H]
\centering
\caption{Uncontrolled, Matching, and Regression Estimates of the Effects of Voluntary Military Service on Earnings}
\scriptsize
\begin{tabular}{lccccc}
\hline
Race & \shortstack{Average \\ earnings \\ 1988--1991 \\ (1)} 
     & \shortstack{Diff. in means \\ by veteran status \\ (2)} 
     & \shortstack{Matching \\ estimates \\ (3)} 
     & \shortstack{Regression \\ estimates \\ (4)} 
     & \shortstack{Regression \\ minus matching \\ (5)} \\
\hline
Whites    & 14537 & 1233.4  & -197.2  & -88.8   & 108.4 \\
          & (60.3) & (60.3) & (70.5)  & (62.5)  & (28.5) \\
Nonwhites & 11664 & 2449.1  & 839.7   & 1074.4  & 234.7 \\
          & (47.4) & (47.4) & (62.7)  & (50.7)  & (32.5) \\
\hline
\end{tabular}

\begin{flushleft}
\footnotesize 
Notes: Adapted from Angrist (1998, Tables II and V). Standard errors in parentheses.  
Estimates cover 1988--1991 Social Security-taxable earnings of men who applied to enter the armed forces between 1979 and 1982.  
Controls include year of birth, education at application, and AFQT score.  
Sample sizes: 128,968 whites and 175,262 nonwhites.  
\end{flushleft}
\end{table}

\doublespacing
\textbf{Explanation} \\
Table 3 summarizes the estimated effects of \textbf{voluntary military service} 
on 1988--1991 earnings. Column (1) shows the average earnings for whites and nonwhites.  
Column (2) presents the \textbf{raw differences in means} between veterans and nonveterans.  
Columns (3) and (4) report \textbf{matching} and \textbf{regression} estimates, 
which adjust for birth year, education, and AFQT scores.  
Column (5) shows the difference between regression and matching estimates.  
Standard errors are in parentheses.  

\singlespacing
\textbf{Intuition}
\begin{itemize}
    \item Raw differences suggest veterans earn \textbf{more}, especially among nonwhites
    \item Once we \textbf{control for background factors}, the advantage for whites 
    \textbf{disappears}, but remains \textbf{positive and large} for nonwhites
    \item Think of it like comparing two classrooms: if we just take the average grades, 
    veterans seem ahead. But once we compare students with the \textbf{same prior preparation}, 
    the gap mostly closes for whites, while nonwhites still keep a real boost
\end{itemize}

\section*{\noindent\textbf{Regression vs. Matching}}
\addcontentsline{toc}{section}{Regression vs. Matching}

\begin{itemize}
    \item Angrist (1998) reports estimates of \(\delta_R\) in the regression:
\end{itemize}

\singlespacing
\begin{align}
Y_i &= \sum_x d_{ix}\beta_x + \delta_R D_i + \varepsilon_i 
   & \text{\textbf{\shortstack{Model with \\ covariates}}}
\end{align}

\begin{itemize}
    \item where \(d_{ix}\) indicates \(X_i = x\), \(\beta_x\) is the regression effect for cell \(X_i=x\),  
    and \(\delta_R\) is the treatment effect from regression.
\end{itemize}

\begin{itemize}
    \item Simplifying, the OLS estimator is:
\end{itemize}

\singlespacing
\begin{align}
\delta_R &= \frac{\text{Cov}(Y_i, \tilde{D}_i)}{\text{Var}(\tilde{D}_i)} 
   & \text{\textbf{\shortstack{OLS \\ formula}}} \\[6pt]
\tilde{D}_i &= D_i - \mathbb{E}[D_i \mid X_i] 
   & \text{\textbf{\shortstack{Residualized \\ treatment}}}
\end{align}

\singlespacing
\begin{align}
\text{Cov}(Y_i, \tilde{D}_i) 
   &= \mathbb{E}[(Y_i - \mathbb{E}[Y_i]) (D_i - \mathbb{E}[D_i \mid X_i])] 
   & \text{\textbf{\shortstack{Def. of \\ Cov}}} \\[6pt]
   &= \mathbb{E}[(D_i - \mathbb{E}[D_i \mid X_i]) Y_i] 
   & \text{\textbf{\shortstack{Cross-term \\ simplification}}}
\end{align}

\singlespacing
\begin{align}
\delta_R 
   &= \frac{\mathbb{E}[(D_i - \mathbb{E}[D_i \mid X_i])Y_i]}{\mathbb{E}[(D_i - \mathbb{E}[D_i \mid X_i])^2]} 
   & \text{\textbf{\shortstack{Plug into \\ OLS ratio}}} \\[6pt]
   &= \frac{\mathbb{E}\{(D_i - \mathbb{E}[D_i \mid X_i]) \cdot \mathbb{E}[Y_i \mid D_i, X_i]\}}{\mathbb{E}[(D_i - \mathbb{E}[D_i \mid X_i])^2]} 
   & \text{\textbf{\shortstack{Use law of \\ iterated exp.}}}
\end{align}

\begin{itemize}
    \item Interpretation:
\end{itemize}

\singlespacing
\begin{align}
\delta_R 
   &= \frac{\text{Weighted average of conditional treatment effects}}{\text{Variance of residualized $D_i$}} 
   & \text{\textbf{\shortstack{Interpretation \\ of estimator}}}
\end{align}

\textbf{Intuition}
\begin{itemize}
    \item Regression with controls is like comparing veterans and nonveterans 
    \textbf{after “residualizing” treatment} on covariates
    \item Matching does the same idea more directly: build \textbf{pairs of similar units} and compare
    \item Example: Suppose age predicts enlistment. Regression subtracts the “age effect” before comparing earnings; matching compares \textbf{same-age individuals} directly
\end{itemize}

\section*{\noindent\textbf{Regression vs. Matching II}}
\addcontentsline{toc}{section}{Regression vs. Matching II}

\begin{itemize}
    \item Start from the conditional expectation of $Y_i$ given $D_i$ and $X_i$:
\end{itemize}

\singlespacing
\begin{align}
\mathbb{E}[Y_i \mid D_i, X_i] 
   &= \mathbb{E}[Y_i \mid D_i=0, X_i] + \delta_{X} D_i 
   & \text{\textbf{\shortstack{Treatment \\ effect by $X$}}}
\end{align}

\begin{itemize}
    \item Substitute this into the numerator of $\delta_R$:
\end{itemize}

\singlespacing
\begin{align}
\mathbb{E}\{ (D_i - \mathbb{E}[D_i \mid X_i]) Y_i \} 
   &= \mathbb{E}\{ (D_i - \mathbb{E}[D_i \mid X_i]) \cdot \mathbb{E}[Y_i \mid D_i, X_i] \} 
   & \text{\textbf{\shortstack{Replace $Y_i$ \\ with cond. exp.}}} \\[6pt]
&= \mathbb{E}\{ (D_i - \mathbb{E}[D_i \mid X_i]) [\mathbb{E}[Y_i \mid D_i=0, X_i] + \delta_X D_i] \} 
   & \text{\textbf{\shortstack{Expand using \\ formula above}}}
\end{align}

\singlespacing
\begin{align}
&= \mathbb{E}\{ (D_i - \mathbb{E}[D_i \mid X_i]) \cdot \mathbb{E}[Y_i \mid D_i=0, X_i] \} 
   + \mathbb{E}\{ (D_i - \mathbb{E}[D_i \mid X_i]) \cdot \delta_X D_i \} 
   & \text{\textbf{Distribute terms}} \\[6pt]
&= 0 + \mathbb{E}\{ (D_i - \mathbb{E}[D_i \mid X_i]) D_i \delta_X \} 
   & \text{\textbf{\shortstack{Uncorrelated with \\ baseline term}}}
\end{align}

\singlespacing
\begin{align}
&= \mathbb{E}\{ (D_i - \mathbb{E}[D_i \mid X_i]) D_i \delta_X \} 
   & \text{\textbf{Keep treatment part}} \\[6pt]
&= \mathbb{E}\{ (D_i - \mathbb{E}[D_i \mid X_i])^2 \delta_X \} 
   & \text{\textbf{\shortstack{Because factor $D_i$ \\ multiplies residual}}}
\end{align}

\begin{itemize}
    \item Therefore:
\end{itemize}

\singlespacing
\begin{align}
\delta_R 
   &= \frac{\mathbb{E}\{ (D_i - \mathbb{E}[D_i \mid X_i])^2 \delta_X \}}{\mathbb{E}[(D_i - \mathbb{E}[D_i \mid X_i])^2]} 
   & \text{\textbf{\shortstack{Plug into \\ OLS ratio}}}
\end{align}

\begin{itemize}
    \item Iterate expectations over $X$:
\end{itemize}

\singlespacing
\begin{align}
\delta_R 
   &= \frac{\mathbb{E}\big[ \mathbb{E}[(D_i - \mathbb{E}[D_i \mid X_i])^2 \mid X_i] \cdot \delta_X \big]}{\mathbb{E}\big[\mathbb{E}[(D_i - \mathbb{E}[D_i \mid X_i])^2 \mid X_i]\big]} 
   & \text{\textbf{\shortstack{Law of Iterated \\ Expectations}}}
\end{align}

\singlespacing
\begin{align}
\delta_R 
   &= \frac{\mathbb{E}[ \sigma_D^2(X_i) \cdot \delta_X ]}{\mathbb{E}[\sigma_D^2(X_i)]} 
   & \text{\textbf{\shortstack{Define \\ conditional variance}}}
\end{align}

where \(\sigma_D^2(X_i) = \mathbb{E}[(D_i - \mathbb{E}[D_i \mid X_i])^2 \mid X_i]\).

\textbf{Intuition}
\begin{itemize}
    \item Regression puts more weight on cells where $D_i$ has higher \textbf{variance within $X_i$} (more treated and untreated units to compare)
    \item Matching weights all cells more evenly, regression emphasizes those with \textbf{better overlap}
    \item Example: If veterans and nonveterans are balanced in education group A but rare in group B, regression leans on group A for precision, while matching treats both groups similarly
\end{itemize}

\section*{\noindent\textbf{Regression vs. Matching III}}
\addcontentsline{toc}{section}{Regression vs. Matching III}

\textbf{Step 1. Variance of $D_i$ conditional on $X_i$}

Since $D_i$ is binary:
\[
\sigma_D^2(X_i) = \mathbb{E}[(D_i - \mathbb{E}[D_i|X_i])^2 \mid X_i]
\]

But $\mathbb{E}[D_i|X_i] = P(D_i=1 \mid X_i)$.  
So,
\[
\sigma_D^2(X_i) = P(D_i=1|X_i) \cdot (1-P(D_i=1|X_i))
\]

\textbf{Step 2. Regression estimand as weighted average}

We had:
\[
\delta_R = \frac{\mathbb{E}[\sigma_D^2(X_i)\delta_X]}{\mathbb{E}[\sigma_D^2(X_i)]}
\]

Expanding over support of $X$:
\[
\delta_R = \frac{\sum_x \delta_x \, \sigma_D^2(x) \, P(X_i=x)}{\sum_x \sigma_D^2(x) \, P(X_i=x)}
\]

Explicitly:
\[
\delta_R = \frac{\sum_x \delta_x \, [P(D_i=1|X_i=x)(1-P(D_i=1|X_i=x))] P(X_i=x)}{\sum_x [P(D_i=1|X_i=x)(1-P(D_i=1|X_i=x))] P(X_i=x)}
\]

\hfill \textbf{\shortstack{Regression = weighted avg. \\ weights $\propto$ var$(D|X)$}}

\textbf{Step 3. Matching / TOT definition}

The treatment-on-the-treated (TOT) effect:
\[
\mathbb{E}[Y_{1i} - Y_{0i} \mid D_i=1] = \sum_x \delta_x \, P(X_i=x \mid D_i=1)
\]

\textbf{Step 4. Expand conditional probability}

By Bayes rule:
\[
P(X_i=x \mid D_i=1) = \frac{P(D_i=1 \mid X_i=x) P(X_i=x)}{P(D_i=1)}
\]

So,
\[
\text{TOT} = \frac{\sum_x \delta_x P(D_i=1 \mid X_i=x) P(X_i=x)}{\sum_x P(D_i=1 \mid X_i=x) P(X_i=x)}
\]

\hfill \textbf{\shortstack{TOT = weighted avg. \\ weights $\propto P(D=1|X)P(X)$}}


\textbf{Step 5. Comparison}

- Regression estimator $\delta_R$ weights by \textit{overlap/variance}: $P(D=1|X)(1-P(D=1|X))P(X)$.  
- TOT weights by \textit{treated mass}: $P(D=1|X)P(X)$.  

\textbf{Intuition}

\begin{itemize}
    \item \textbf{Regression}: emphasizes strata $x$ where treatment is more \underline{balanced}, since variance of $D$ is maximized when $P(D=1|X)=0.5$.
    \item \textbf{TOT}: emphasizes strata $x$ where more people are treated, regardless of balance.
    \item If veterans are spread across education groups, regression will lean more on groups with mix of veterans and non-veterans, while TOT leans more on groups with many veterans (even if imbalance is large).
\end{itemize}

\section*{\noindent\textbf{Regresión vs. Matching IV}}
\addcontentsline{toc}{section}{Regresión vs. Matching IV}

\begin{itemize}
    \item \textbf{TOT}: pondera cada celda de covariantes $x$ en proporción a la \textit{probabilidad de tratamiento}, $P(D_i=1|X_i=x)P(X_i=x)$.
    \item \textbf{Regresión}: pondera cada celda en proporción a la \textit{varianza condicional del tratamiento}, 
    \[
    P(D_i=1|X_i=x)\big(1-P(D_i=1|X_i=x)\big)P(X_i=x).
    \]
    \item Esta varianza se maximiza cuando $P(D_i=1|X_i=x)=\tfrac{1}{2}$ (máximo balance).
\end{itemize}

\textbf{Ejemplo breve:}  
\begin{itemize}
    \item Supongamos 2 grupos $X=A,B$ con efectos $\delta_A=1000$ y $\delta_B=2000$.
    \item En $A$, 90\% tratados; en $B$, 50\% tratados.
    \item TOT da más peso al grupo $A$ (muchos tratados) $\Rightarrow$ estimador $\approx 1000$.
    \item Regresión da más peso al grupo $B$ (más balance) $\Rightarrow$ estimador $\approx 2000$.
\end{itemize}

\textbf{Intuición}:  
\begin{itemize}
    \item \textbf{TOT} sigue a los tratados: mide “lo que realmente experimentaron”.
    \item \textbf{Regresión} sigue al balance: se concentra en los grupos donde la comparación tratado/no tratado es más creíble.
\end{itemize}

\begin{figure}[H]
    \centering
    \includegraphics[width=0.8\textwidth]{image3}
    \caption{\footnotesize Controlled contrasts by race and probability of service. 
    These estimates are for pooled 1988--91 earnings.}
    \label{fig:contrasts_prob_service}
\end{figure}

\begin{itemize}
    \item \textbf{Explanation:} The earnings gap is large and positive when the probability of service is low, but it shrinks and even disappears as the probability rises. This reflects selection: early entrants differ from later ones. 
    \item \textbf{Intuition:} At first, only the most motivated or skilled joined, so veterans look richer. As more people serve, including less advantaged groups, the average veteran advantage fades.
\end{itemize}

\section*{\noindent\textbf{References}}
\addcontentsline{toc}{section}{References}

\begin{itemize}
    \item \textbf{Angrist, Joshua D.} and \textbf{Joern-Steffen Pischke} (2009). 
    \textit{Mostly Harmless Econometrics: An Empiricist’s Companion}, 
    Princeton, NJ: Princeton University Press. Section 3.3.1.

    \item \textbf{Angrist, Joshua D.} (1998). 
    ``Estimating the Labor Market Impact of Voluntary Military Service using Social Security Data on Military Applicants.'' 
    \textit{Econometrica}, 66(2): 249--288.
\end{itemize}



\end{document}

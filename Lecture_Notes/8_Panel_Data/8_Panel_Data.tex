\documentclass[12pt]{article}

% --- Paquetes ---
\usepackage{pifont} 
\usepackage{tikz}
\usetikzlibrary{arrows.meta}
\usepackage{pgfplots}
\pgfplotsset{compat=1.18}
\usepackage[most]{tcolorbox}
\usepackage[spanish,es-tabla]{babel}   % español
\usepackage[utf8]{inputenc}            % acentos
\usepackage[T1]{fontenc}
\usepackage{lmodern}
\usepackage{geometry}
\usepackage{fancyhdr}
\usepackage{xcolor}
\usepackage{titlesec}
\usepackage{lastpage}
\usepackage{amsmath,amssymb}
\usepackage{enumitem}
\usepackage[table]{xcolor} % para \cellcolor y \rowcolor
\usepackage{colortbl}      % colores en tablas
\usepackage{float}         % para usar [H] si quieres fijar la tabla
\usepackage{array}         % mejor control de columnas
\usepackage{amssymb}       % para palomita
\usepackage{graphicx}      % para logo github
\usepackage{hyperref}
\usepackage{setspace} % para hipervinculo
\usepackage[normalem]{ulem}
\usepackage{siunitx}       % Asegúrate de tener este paquete en el preámbulo
\usepackage{booktabs}
\sisetup{
    output-decimal-marker = {.},
    group-separator = {,},
    group-minimum-digits = 4,
    detect-all
}

% Etiqueta en el caption (en la tabla misma)
\usepackage{caption}
\captionsetup[table]{name=Tabla, labelfont=bf, labelsep=period}

% Prefijo en la *Lista de tablas*
\usepackage{tocloft}
\renewcommand{\cfttabpresnum}{Tabla~} % texto antes del número
\renewcommand{\cfttabaftersnum}{.}    % punto después del número
\setlength{\cfttabnumwidth}{5em}      % ancho para "Tabla 10." ajusta si hace falta



% --- Márgenes y encabezado ---
\geometry{left=1in, right=1in, top=1in, bottom=1in}

% Alturas del encabezado (un poco más por las 2–3 líneas del header)
\setlength{\headheight}{32pt}
\setlength{\headsep}{20pt}

\definecolor{maroon}{RGB}{128, 0, 0}

\pagestyle{fancy}
\fancyhf{}

% Regla del encabezado (opcional)
\renewcommand{\headrulewidth}{0.4pt}

% Encabezado izquierdo
\fancyhead[L]{%
  \textcolor{maroon}{\textbf{El Colegio de México}}\\
  \textbf{Microeconometrics for Evaluation}
}

% Encabezado derecho
\fancyhead[R]{%
  8 Panel Data\\
  \textbf{Jose Daniel Fuentes García}\\
  Github : \includegraphics[height=1em]{github.png}~\href{https://github.com/danifuentesga}{\texttt{danifuentesga}}
}

% Número de página al centro del pie
\fancyfoot[C]{\thepage}

% --- APLICAR EL MISMO ESTILO A PÁGINAS "PLAIN" (TOC, LOT, LOF) ---
\fancypagestyle{plain}{%
  \fancyhf{}
  \renewcommand{\headrulewidth}{0.4pt}
  \fancyhead[L]{%
    \textcolor{maroon}{\textbf{El Colegio de México}}\\
    \textbf{Microeconometrics for Evaluation}
  }
  \fancyhead[R]{%
  8 Panel Data\\
    \textbf{Jose Daniel Fuentes García}\\
    Github : \includegraphics[height=1em]{github.png}~\href{https://github.com/danifuentesga}{\texttt{danifuentesga}}
  }
  \fancyfoot[C]{\thepage}
}

% Pie de página centrado
\fancyfoot[C]{\thepage\ de \pageref{LastPage}}

\renewcommand{\headrulewidth}{0.4pt}

% --- Color principal ---
\definecolor{formalblue}{RGB}{0,51,102} % azul marino sobrio

% --- Estilo de títulos ---
\titleformat{\section}[hang]{\bfseries\Large\color{formalblue}}{}{0em}{}[\titlerule]
\titleformat{\subsection}{\bfseries\large\color{formalblue}}{\thesubsection}{1em}{}


% --- Listas ---
\setlist[itemize]{leftmargin=1.2em}

% --- Sin portada ---
\title{}
\author{}
\date{}

\begin{document}

\begin{titlepage}
    \vspace*{-1cm}
    \noindent
    \begin{minipage}[t]{0.49\textwidth}
        \includegraphics[height=2.2cm]{colmex.jpg}
    \end{minipage}%
    \begin{minipage}[t]{0.49\textwidth}
        \raggedleft
        \includegraphics[height=2.2cm]{cee.jpg}
    \end{minipage}

    \vspace*{2cm}

    \begin{center}
        \Huge \textbf{CENTRO DE ESTUDIOS ECONÓMICOS} \\[1.5em]
        \Large Maestría en Economía 2024--2026 \\[2em]
        \Large Microeconometrics for Evaluation \\[3em]
        \LARGE \textbf{8 Panel Data} \\[3em]
        \large \textbf{Disclaimer:} I AM NOT the original intellectual author of the material presented in these notes. The content is STRONGLY based on a combination of lecture notes (from Aurora Ramirez), textbook references, and personal annotations for learning purposes. Any errors or omissions are entirely my own responsibility.\\[0.9em]
        
    \end{center}

    \vfill
\end{titlepage}

\newpage

\setcounter{secnumdepth}{2}
\setcounter{tocdepth}{4}
\tableofcontents

\newpage

\section*{\noindent\textbf{Datos de panel}}
\addcontentsline{toc}{section}{Datos de panel}

\begin{itemize}
    \item Los datos de panel incluyen observaciones de:
    \begin{itemize}
        \item Las mismas unidades (por ejemplo, individuos, empresas, países, ...) 
        que se observan durante al menos dos periodos de tiempo (años, meses, días, ...).
    \end{itemize}
    
    \item Los conjuntos de datos de panel vienen en dos formas:
    \begin{itemize}
        \item \textbf{Panel balanceado:} cada unidad se observa durante el mismo periodo de tiempo.
        \item \textbf{Panel no balanceado:} se observan unidades para distintos periodos de tiempo.
    \end{itemize}
\end{itemize}

\begin{itemize}
    \item \textbf{Explicación:} Un panel es una base de datos que combina información en dos dimensiones: transversal (diferentes unidades) y temporal (diferentes momentos en el tiempo).
    \item \textbf{Intuición:} Es como tener el historial médico de varios pacientes: 
    un \textit{panel balanceado} sería cuando todos tienen chequeos anuales; 
    un \textit{no balanceado}, cuando algunos pacientes tienen años faltantes en sus registros.
\end{itemize}

\section*{\noindent\textbf{Métodos de datos panel}}
\addcontentsline{toc}{section}{Métodos de datos panel}

\begin{itemize}
    \item A menudo, nuestra variable dependiente depende de factores no observados 
    que también están correlacionados con nuestra variable explicativa de interés.
    
    \item Si estas variables omitidas son constantes en el tiempo, 
    podemos usar los estimadores de datos de panel para estimar 
    el efecto de nuestra variable explicativa.
    
    \item \textbf{Principales estimadores de datos de panel:}
    \begin{enumerate}
        \item MCO agrupados (\textit{pooled OLS})
        \item Métodos de efectos fijos (FE)
        \item Método de efectos aleatorios (RE)
    \end{enumerate}
\end{itemize}

\begin{itemize}
    \item \textbf{Explicación:} Los modelos de panel permiten controlar por heterogeneidad no observada entre individuos que no varía en el tiempo.
    \item \textbf{Intuición:} Es como analizar el desempeño de estudiantes en varias materias a lo largo de los años: 
    los \textit{efectos fijos} eliminan características propias del alumno que no cambian, 
    mientras que los \textit{efectos aleatorios} suponen que esas diferencias son aleatorias.
\end{itemize}

\section*{\noindent\textbf{Modelo de efectos no observados: producción agrícola}}
\addcontentsline{toc}{section}{Modelo de efectos no observados: producción agrícola}

\[
y_{it} = x_{it}\beta + c_i + u_{it}, 
\quad t = 1,2,\dots,T
\]

\begin{itemize}
    \item $y_{it}$: producción de la granja $i$ en el año $t$.
    \item $x_{it}$: vector $1 \times K$ de insumos variables para la granja $i$ en el año $t$ 
    (ej. mano de obra, fertilizante). Incluye la constante.
    \item $\beta$: vector $K \times 1$ de efectos marginales de los insumos variables.
    \item $c_i$: suma de insumos fijos en el tiempo, conocidos por el agricultor 
    pero no observados por el investigador 
    (ej. calidad del suelo, capacidad de gestión). $\rightarrow$ Heterogeneidad no observada.
    \item $u_{it}$: insumos no observados que varían en el tiempo 
    (ej. lluvia, plagas). Error idiosincrático.
    \item \textbf{Pregunta central:} ¿Qué sucede cuando corremos una regresión de $y_{it}$ en $x_{it}$?
\end{itemize}

\begin{itemize}
    \item \textbf{Explicación:} El término $c_i$ puede estar correlacionado con $x_{it}$, 
    lo que genera sesgo en los estimadores MCO si no se controla.
    \item \textbf{Intuición:} Cada granja tiene características fijas (suelo, gestión) que afectan la producción. 
    Si no las controlamos, confundimos su efecto con el de los insumos variables.
\end{itemize}

\section*{\noindent\textbf{MCO agrupados (pooled OLS)}}
\addcontentsline{toc}{section}{MCO agrupados (pooled OLS)}

\[
y_{it} = x_{it}\beta + v_{it}, \quad t = 1,2,\dots,T
\]
con \quad $v_{it} = c_i + u_{it}$

\begin{itemize}
    \item Ignoramos la estructura de panel y tratamos todos los datos como una muestra agrupada.
    \item Supuestos clave para consistencia:
    \begin{enumerate}
        \item $E(x_{it}^T v_{it}) = 0, \quad t=1,2,\dots,T$
        \item $rank\!\left[\sum_{t=1}^T E(x_{it}^T x_{it})\right] = K$
    \end{enumerate}
    \item Bajo estos supuestos, el error compuesto $v_{it}$ no debe estar correlacionado con los regresores.
    \item Problema: si $x_{it}$ depende de características no observadas $c_i$ (ej. calidad del suelo), 
    aparece sesgo de selección.
    \item Entonces, $\hat{\beta}_{OLS}$ no es consistente.
\end{itemize}

\begin{itemize}
    \item \textbf{Explicación:} Pooled OLS mezcla toda la variación sin separar lo fijo ($c_i$) de lo idiosincrático ($u_{it}$).
    \item \textbf{Intuición:} Es como comparar granjas sin reconocer que cada una tiene “suelo distinto” constante en el tiempo. 
    Esa heterogeneidad se cuela en el estimador y sesga los resultados.
\end{itemize}

\hrule

\begin{itemize}
    \item Ninguna correlación entre $x_{it}$ y $v_{it}$ implica también que no haya correlación entre 
    el efecto no observado $c_i$ y $x_{it}$.
    \item Problema adicional: $v_{it}$ están correlacionados en serie para el mismo $i$, 
    porque $c_i$ aparece en todos los periodos $t$.
    \item Consecuencia: los errores estándar de MCO agrupados no son válidos.
\end{itemize}

\begin{itemize}
    \item \textbf{Explicación:} Como $c_i$ se repite para cada individuo, los residuos están correlacionados dentro de cada $i$.
    \item \textbf{Intuición:} Es como medir varias veces la misma granja: el “suelo” no cambia, 
    por lo que los errores están inflados y los intervalos de confianza resultan engañosos.
\end{itemize}

\section*{\noindent\textbf{Modelo de efectos no observados: evaluación de un programa}}
\addcontentsline{toc}{section}{Modelo de efectos no observados: evaluación de un programa}

\[
y_{it} = prog_{it} \beta + c_i + u_{it}, \quad t=1,2,\ldots,T
\]

\begin{itemize}
    \item $y_{it}$: logaritmo del salario del individuo $i$ en el año $t$.
    \item $prog_{it}$: indicador =1 si el individuo $i$ participa en el programa de entrenamiento en $t$, y 0 en caso contrario.
    \item $\beta$: efecto del programa.
    \item $c_i$: suma de todas las características no observadas e invariantes en el tiempo que afectan los salarios (ejemplo: habilidad).
    \item Problema: si corremos una regresión de $y_{it}$ en $prog_{it}$, 
    $\beta$ no es consistente porque $prog_{it}$ probablemente esté correlacionado con $c_i$.
    \item Siempre preguntarse: ¿hay alguna variable no observada constante en el tiempo $c_i$ 
    correlacionada con los regresores?  
    Si la respuesta es sí, los MCO agrupados son problemáticos.
\end{itemize}

\begin{itemize}
    \item \textbf{Explicación:} $c_i$ captura heterogeneidad fija (ejemplo: habilidad), y si está correlacionada con $prog_{it}$, el estimador de MCO queda sesgado.
    \item \textbf{Intuición:} Si los más hábiles tienen más probabilidad de entrar al programa, 
    el efecto estimado confundirá “programa” con “habilidad”.
\end{itemize}

\section*{\noindent\textbf{Supuesto de exogeneidad estricta}}
\addcontentsline{toc}{section}{Supuesto de exogeneidad estricta}

\begin{itemize}
    \item Para estimar los modelos de datos de panel más básicos (efectos fijos y efectos aleatorios), 
    se asume \textbf{estricta exogeneidad}:
    \[
    E(y_{it} \mid x_{i1}, x_{i2}, ..., x_{iT}, c_i) 
    = E(y_{it}\mid x_{it}, c_i) 
    = x_{it}\beta + c_i
    \]
    \item En palabras: una vez que se controla por $x_{it}$ y $c_i$, ninguna otra $x_{is}$ 
    con $s \neq t$ tiene efecto parcial sobre $y_{it}$.
    \item En el modelo 
    \[
    y_{it} = x_{it}\beta + c_i + u_{it},
    \]
    la exogeneidad estricta se expresa como:
    \[
    E(u_{it} \mid x_{i1}, x_{i2},..., x_{iT}, c_i) = 0, 
    \quad t=1,2,\ldots,T
    \]
\end{itemize}

\begin{itemize}
    \item \textbf{Explicación:} Los regresores en cualquier periodo no deben estar correlacionados con el error idiosincrático en ningún periodo.
    \item \textbf{Intuición:} El pasado, presente o futuro de $x_{it}$ no debe “predecir” shocks en $u_{it}$.  
    Ejemplo: si una política se implementa cuando se anticipa una crisis, se viola la exogeneidad estricta.
\end{itemize}

\hrule

\begin{itemize}
    \item Este supuesto implica que las variables explicativas en cada periodo de tiempo 
    no están correlacionadas con el error idiosincrático $u_{it}$ en ningún periodo:
    \[
    E(x_{is}^T u_{it}) = 0, 
    \quad s,t=1,\ldots,T
    \]
    \item Es decir, los regresores en el pasado, presente y futuro son exógenos con respecto a los errores de todos los periodos.
    \item Este supuesto es mucho más fuerte que asumir simplemente \textbf{no correlación contemporánea}:
    \[
    E(x_{it}^T u_{it}) = 0, 
    \quad t=1,\ldots,T
    \]
\end{itemize}

\begin{itemize}
    \item \textbf{Explicación:} Exogeneidad estricta excluye toda forma de retroalimentación o anticipación entre regresores y errores.
    \item \textbf{Intuición:} No basta con que las $x_{it}$ sean “limpias” en el mismo periodo; también deben ser independientes de shocks pasados y futuros.  
    Ejemplo: si los agricultores ajustan la mano de obra porque anticipan lluvias (shock futuro), se rompe la exogeneidad estricta.
\end{itemize}

\section*{\noindent\textbf{Efectos Fijos (FE)}}
\addcontentsline{toc}{section}{Efectos Fijos (FE)}

\begin{itemize}
    \item Los efectos fijos abordan explícitamente el hecho de que $c_i$ puede correlacionarse con $x_{it}$.
    \item Para los modelos de efectos fijos asumimos \textbf{estricta exogeneidad}:
    \[
    FE1: \quad E(u_{it} \mid x_i, c_i) = 0, \quad t=1,2,\ldots,T
    \]
    \item Donde $x_i = (x_{i1},x_{i2},\ldots,x_{iT})$
    \item Se permite que $E(c_i \mid x_i)$ sea cualquier función de $x_i$.
    \item \textbf{Costo:} no podemos incluir variables constantes en el tiempo en $x_{it}$ (por ejemplo, género, región fija).
\end{itemize}

\begin{itemize}
    \item \textbf{Explicación:} FE elimina la heterogeneidad inobservable $c_i$ controlando por su correlación con los regresores.
    \item \textbf{Intuición:} Comparamos cada unidad consigo misma a lo largo del tiempo. 
    Ejemplo: si una escuela siempre tiene buena infraestructura ($c_i$), los FE eliminan ese efecto fijo y usan solo cambios dentro de la escuela.
\end{itemize}

\section*{\noindent\textbf{Efectos fijos - Formas de eliminar $c_i$}}
\addcontentsline{toc}{section}{Efectos fijos - Formas de eliminar $c_i$}

\begin{itemize}
    \item En los modelos \textbf{FE} existen tres maneras de remover el término $c_i$, 
    que genera \textbf{correlación} entre el error y los regresores.
    \item \textbf{Opciones:}
    \begin{itemize}
        \item \textbf{Within-transformation} (\textbf{transformación FE}).
        \item Estimación de $c_i$ mediante \textbf{dummies}.
        \item Uso de \textbf{primeras diferencias}.
    \end{itemize}
\end{itemize}

\noindent\textbf{Intuición}
\begin{itemize}
    \item \textbf{Explicación:} Todas las técnicas buscan aislar la variación dentro de los individuos 
    y así evitar la correlación entre $c_i$ y los regresores.
    \item \textbf{Ejemplo:} Si seguimos a una persona en el tiempo, restar su propia media o usar su diferencia 
    entre dos periodos elimina características fijas como su talento innato.
\end{itemize}

\subsection*{\noindent\textbf{1 Within Estimator}}
\addcontentsline{toc}{subsection}{1 Within Estimator}

\begin{itemize}
    \item \textbf{Ecuación a estimar:} 
    \[
    y_{it} = x_{it}\beta + c_i + u_{it} \tag{1}
    \]
    
    \item \textbf{Paso 1:} Promediar la ecuación (1) sobre $t = 1, \ldots, T$:
    \[
    \bar{y}_i = \bar{x}_i \beta + c_i + \bar{u}_i \tag{2}
    \]
    donde:
    \[
    \bar{y}_i = \frac{1}{T}\sum_{t=1}^T y_{it}, \quad 
    \bar{x}_i = \frac{1}{T}\sum_{t=1}^T x_{it}, \quad 
    \bar{u}_i = \frac{1}{T}\sum_{t=1}^T u_{it}
    \]
    
    \item \textbf{Paso 2:} Restar la ecuación (2) de (1) para obtener:
    \[
    y_{it} - \bar{y}_i = (x_{it} - \bar{x}_i)\beta + (u_{it} - \bar{u}_i)
    \]
    \[
    \tilde{y}_{it} = \tilde{x}_{it}\beta + \tilde{u}_{it}
    \]
    donde: 
    \[
    \tilde{y}_{it} = y_{it} - \bar{y}_i, \quad 
    \tilde{x}_{it} = x_{it} - \bar{x}_i, \quad 
    \tilde{u}_{it} = u_{it} - \bar{u}_i
    \]
    
    \item \textbf{Paso 3:} Estimar una regresión de $\tilde{y}_{it}$ sobre $\tilde{x}_{it}$ usando \textbf{MCO agrupados}.
\end{itemize}

\noindent\textbf{Intuición}
\begin{itemize}
    \item \textbf{Explicación:} El método within elimina $c_i$ comparando a cada individuo consigo mismo a lo largo del tiempo.
    \item \textbf{Ejemplo:} Es como medir el cambio de salario de una persona en distintos años, ignorando su talento fijo o características innatas.
\end{itemize}

\hrule

\begin{itemize}
    \item Para que el estimador \textbf{FE} tenga buen comportamiento asintótico, se requiere la \textbf{condición de rango estándar}:  
    \[
    \text{FE2: } \quad \text{rank}\!\left(\sum_{t=1}^T E(\tilde{x}_{it}\tilde{x}_{it}^T)\right) = K
    \]
    
    \item Si $x_{it}$ incluye un componente que no cambia en el tiempo para cada $i$, entonces su correspondiente en $\tilde{x}_{it}$ será \textbf{cero}, y la condición de rango no se cumplirá.
    
    \item Esto implica que no se pueden incluir \textbf{variables invariantes en el tiempo} en modelos de efectos fijos.
    
    \item Además, sin supuestos adicionales, el estimador \textbf{FE} no es necesariamente el más \textbf{eficiente}.  
    El siguiente supuesto asegura eficiencia y una matriz de varianza apropiada:  
    \[
    \text{FE3: } \quad E(u_i u_i^T \,|\, x_i, c_i) = \sigma_u^2 I_T
    \]
\end{itemize}

\noindent\textbf{Intuición}
\begin{itemize}
    \item \textbf{Explicación:} Las condiciones FE2 y FE3 garantizan que el estimador FE pueda identificar correctamente los parámetros y obtener varianzas válidas.
    \item \textbf{Ejemplo:} Si tratamos de usar género en un modelo FE, siempre será constante para cada individuo y no se podrá estimar su efecto.
\end{itemize}

\subsection*{\noindent\textbf{Within Estimator: Errores estándar}}
\addcontentsline{toc}{subsection}{Within Estimator: Errores estándar}

\begin{itemize}
    \item Los errores estándar de la regresión \textbf{MCO} en el paso 3 no son correctos.  
    \item ¿Por qué ocurre? Al restar la media a cada observación (\textbf{demeaning}) se introduce \textbf{correlación serial} en los errores.
    
    \item \textbf{Varianza de $\tilde{u}_{it}$}:  
    \[
    E(\tilde{u}_{it}^2) = E[(u_{it} - \bar{u}_i)^2] 
    = E(u_{it}^2) + E(\bar{u}_i^2) - 2E(u_{it}\bar{u}_i)
    \]
    \[
    = \sigma_u^2 + \frac{\sigma_u^2}{T} - \frac{2\sigma_u^2}{T} 
    = \sigma_u^2 \left(1 - \frac{1}{T}\right)
    \]
    
    \item \textbf{Covarianza entre $\tilde{u}_{it}$ y $\tilde{u}_{is}$ para $t \neq s$}:  
    \[
    E(\tilde{u}_{it}\tilde{u}_{is}) = E[(u_{it} - \bar{u}_i)(u_{is} - \bar{u}_i)]
    \]
    \[
    = E(u_{it}u_{is}) - E(u_{it}\bar{u}_i) - E(u_{is}\bar{u}_i) + E(\bar{u}_i^2)
    \]
    \[
    = 0 - \frac{\sigma_u^2}{T} - \frac{\sigma_u^2}{T} + \frac{\sigma_u^2}{T} 
    = -\frac{\sigma_u^2}{T}
    \]
\end{itemize}

\noindent\textbf{Intuición}
\begin{itemize}
    \item \textbf{Explicación:} El proceso de restar medias genera correlación artificial entre los errores dentro del mismo individuo, lo cual sesga los errores estándar simples.
    \item \textbf{Ejemplo:} Es como calcular la desviación de cada examen respecto al promedio de un estudiante: esas desviaciones no son independientes entre sí.
\end{itemize}

\hrule

\begin{itemize}
    \item Como resultado, la \textbf{correlación} entre $\tilde{u}_{it}$ y $\tilde{u}_{is}$ es:
    \[
    \text{Corr}(\tilde{u}_{it}, \tilde{u}_{is}) 
    = \frac{\text{Cov}(\tilde{u}_{it}, \tilde{u}_{is})}{\sqrt{\text{Var}(\tilde{u}_{it})\text{Var}(\tilde{u}_{is})}}
    \]
    \[
    = \frac{-\sigma_u^2 / T}{\sigma_u^2 (1 - 1/T)} 
    = -\frac{1}{T-1}
    \]
    
    \item El \textbf{supuesto 3} nos permite derivar el estimador de la varianza asintótica:
    \[
    \text{Avar}(\hat{\beta}_{FE}) 
    = \hat{\sigma}_u^2 \left( \sum_{i=1}^N \sum_{t=1}^T \tilde{x}_{it}\tilde{x}_{it}^T \right)^{-1}
    \]
    
    \item Notar que $\hat{\sigma}_u^2$ es un \textbf{estimador consistente} para la varianza de $u_{it}$, no de $\tilde{u}_{it}$.
    
    \item Por lo tanto, no es posible obtener $\hat{\sigma}_u^2$ directamente de la regresión MCO del paso 3.
\end{itemize}

\noindent\textbf{Intuición}
\begin{itemize}
    \item \textbf{Explicación:} La correlación negativa entre los errores transformados surge del demeaning, y por eso debemos ajustar la estimación de la varianza.
    \item \textbf{Ejemplo:} Es como medir notas relativas al promedio: si una observación está arriba de la media, otra debe estar abajo, generando correlación automática.
\end{itemize}

\hrule

\begin{itemize}
    \item El \textbf{estimador estándar} de la varianza en la regresión del paso 3 es:
    \[
    \frac{SSR}{NT - K}
    \]
    
    \item Sin embargo, este es un \textbf{estimador incorrecto}, pues necesitamos la varianza de $\hat{\sigma}_u$ y no la de $\hat{\tilde{\sigma}}_u$.
    
    \item La resta de $K$ en el denominador no afecta asintóticamente,  
    pero es habitual aplicar esa \textbf{corrección} en la práctica.
\end{itemize}

\noindent\textbf{Intuición}
\begin{itemize}
    \item \textbf{Explicación:} El error estándar calculado directamente tras el paso 3 no refleja la varianza verdadera de los errores originales, por lo que debe ajustarse.
    \item \textbf{Ejemplo:} Es como calcular una desviación típica con datos transformados: el ajuste asegura que la medida corresponda a la variabilidad real.
\end{itemize}

\hrule

\begin{itemize}
    \item Recordemos que la varianza de $\tilde{u}_{it}$ es:  
    \[
    \sigma_u^2 (1 - 1/T)
    \]
    Sumando sobre $t$ obtenemos:
    \[
    \sum_{t=1}^T E(\tilde{u}_{it}^2) = \sigma_u^2 (T-1)
    \]

    \item Si además sumamos sobre todos los individuos $N$:
    \[
    \sum_{i=1}^N \sum_{t=1}^T E(\tilde{u}_{it}^2) 
    = \sigma_u^2 (T-1)N
    \quad \Rightarrow \quad
    \sigma_u^2 = 
    \frac{\sum_{i=1}^N \sum_{t=1}^T E(\tilde{u}_{it}^2)}{(T-1)N}
    \]

    \item Así, podemos obtener un \textbf{estimador consistente} de $\sigma_u^2$ a partir de la regresión en el paso 3:  
    \[
    \hat{\sigma}_u^2 = \frac{SSR}{N(T-1) - K}
    \]

    \item La diferencia con el uso de $SSR/(NT-K)$ es pequeña, pero este ajuste asegura la correcta varianza bajo el modelo within.
\end{itemize}

\noindent\textbf{Intuición}
\begin{itemize}
    \item \textbf{Explicación:} Al eliminar un grado de libertad por individuo al restar la media, el denominador se ajusta de $NT$ a $N(T-1)$.
    \item \textbf{Ejemplo:} Es como calcular la varianza dentro de cada grupo: al perder una observación efectiva por grupo, se corrige el divisor total.
\end{itemize}

\hrule

\begin{itemize}
    \item Los paquetes de regresión habituales (como \textbf{STATA}) hacen el \textbf{ajuste de errores estándar} de manera automática cuando se especifica un modelo de efectos fijos.
    
    \item Si se desea estimar el modelo \textbf{paso a paso}, se pueden aplicar los tres pasos descritos y luego corregir los errores estándar de la regresión obtenida en el paso 3.
    
    \item El ajuste se logra multiplicando los errores estándar por el factor:
    \[
    \left(\frac{NT - K}{N(T-1) - K}\right)^{1/2}
    \]
\end{itemize}

\noindent\textbf{Intuición}
\begin{itemize}
    \item \textbf{Explicación:} El ajuste corrige la diferencia entre usar $NT$ y $N(T-1)$ observaciones efectivas en el cálculo de varianza.
    \item \textbf{Ejemplo:} Es como estandarizar notas con un divisor ligeramente distinto: el software lo hace automático, pero manualmente hay que escalar los resultados.
\end{itemize}

\section*{\noindent\textbf{2 Estimador de Variables Dummy}}
\addcontentsline{toc}{section}{2 Estimador de Variables Dummy}

\begin{itemize}
    \item Una forma alternativa de estimar modelos de \textbf{efectos fijos} (en particular cuando $N$ es pequeño o si se desean los efectos fijos explícitos) es estimar $c_i$ mediante un conjunto de \textbf{dummies} para cada individuo.
    
    \item Esto implica incluir $N$ dummies (una por cada $i$ en la muestra) dentro de la regresión y estimar:
    \[
    y_{it} = x_{it}\beta + c_i + u_{it} \tag{3}
    \]
    usando \textbf{MCO}. A este procedimiento se le llama \textbf{estimador de variables dummy}.
    
    \item Una ventaja de este método es que entrega los \textbf{errores estándar correctos}, ya que emplea 
    \[
    NT - N - K = N(T-1) - K
    \]
    grados de libertad.
    
    \item El costo es que, si $N$ es grande, la regresión requiere un \textbf{alto poder computacional}.
\end{itemize}

\noindent\textbf{Intuición}
\begin{itemize}
    \item \textbf{Explicación:} Con las dummies se controla directamente por $c_i$, permitiendo estimar efectos fijos sin transformar los datos.
    \item \textbf{Ejemplo:} Es como dar a cada persona un “identificador único” en la regresión, lo que elimina la influencia de sus características permanentes.
\end{itemize}

\section*{\noindent\textbf{3 Primeras Diferencias}}
\addcontentsline{toc}{section}{3 Primeras Diferencias}

\begin{itemize}
    \item Otra alternativa para estimar modelos de \textbf{efectos fijos} es usar las \textbf{primeras diferencias}.
    
    \item Bajo el supuesto de \textbf{exogeneidad estricta} condicional a $c_i$:  
    \[
    \text{FD1: } \; E(u_{it} \mid x_i, c_i) = 0, \quad t = 1,2,\ldots,T
    \]
    
    \item Rezagando el modelo original 
    \[
    y_{it} = x_{it}\beta + c_i + u_{it}
    \]
    un periodo y restando, se obtiene:
    \[
    y_{it} - y_{i,t-1} = (x_{it} - x_{i,t-1})\beta + (c_i - c_i) + (u_{it} - u_{i,t-1})
    \]
    \[
    \Delta y_{it} = \Delta x_{it}\beta + \Delta u_{it}
    \]
    
    \item Tomar \textbf{primeras diferencias} elimina $c_i$ automáticamente.
    
    \item Como consecuencia, se pierde la primera observación temporal de cada unidad de corte transversal.
\end{itemize}

\noindent\textbf{Intuición}
\begin{itemize}
    \item \textbf{Explicación:} Al restar observaciones consecutivas, desaparecen los efectos invariables en el tiempo, dejando solo la variación relevante.
    \item \textbf{Ejemplo:} Es como analizar cuánto cambió el ingreso de un individuo respecto al año anterior: su talento fijo se cancela al comparar diferencias.
\end{itemize}

\hrule

\begin{itemize}
    \item El \textbf{estimador FD} $\hat{\beta}_{FD}$ se obtiene como el estimador de \textbf{MCO agrupados} de la regresión:
    \[
    \Delta y_{it} \; \text{en} \; \Delta x_{it}
    \]
    
    \item Bajo el supuesto \textbf{FD1}, la estimación por MCO agrupados es \textbf{consistente} e \textbf{insesgada}.
    
    \item Como en el caso within, se requiere una \textbf{condición de rango} para el estimador FD:
    \[
    \text{FD2:} \quad \text{rank}\left(\sum_{t=2}^T E(\Delta x_{it}^T \Delta x_{it})\right) = K
    \]
    
    \item Esta condición también excluye a las \textbf{variables invariantes en el tiempo} y evita la \textbf{multicolinealidad perfecta} entre las variables que sí cambian en el tiempo.
\end{itemize}

\noindent\textbf{Intuición}
\begin{itemize}
    \item \textbf{Explicación:} El estimador FD usa solo la variación de un periodo a otro, y necesita que esa variación aporte información independiente para identificar los parámetros.
    \item \textbf{Ejemplo:} Si todas las personas tuvieran exactamente el mismo cambio en ingresos de un año a otro, no se podría estimar el efecto de las variables explicativas.
\end{itemize}

\subsection*{\noindent\textbf{Primeras Diferencias: Errores estándar}}
\addcontentsline{toc}{subsection}{Primeras Diferencias: Errores estándar}

\begin{itemize}
    \item Bajo los supuestos \textbf{FE1–FE3}, el estimador de efectos fijos (\textbf{within}) es asintóticamente \textbf{eficiente}.
    
    \item Sin embargo, el supuesto \textbf{FE3} establece que los $u_{it}$ no presentan \textbf{correlación serial}.
    
    \item Si en su lugar suponemos que los $\Delta u_{it}$ no están serialmente correlacionados, entonces el estimador de \textbf{Primeras Diferencias (FD)} sería \textbf{eficiente}.
    
    \item Además, puede demostrarse que cuando solo existen \textbf{dos periodos}, el estimador FD y el estimador FE son \textbf{idénticos}.
\end{itemize}

\noindent\textbf{Intuición}
\begin{itemize}
    \item \textbf{Explicación:} La eficiencia del estimador depende de cómo se comporten los errores en el tiempo: si la correlación está en $u_{it}$ favorece FE, si está en $\Delta u_{it}$ favorece FD.
    \item \textbf{Ejemplo:} Con solo dos años de datos, calcular diferencias o restar la media es lo mismo: en ambos casos se elimina el efecto fijo $c_i$.
\end{itemize}

\section*{\noindent\textbf{Regresión de efectos fijos: puntos principales}}
\addcontentsline{toc}{section}{Regresión de efectos fijos: puntos principales}

\begin{itemize}
    \item En \textbf{inferencia}, los errores estándar deben estar \textbf{agrupados por unidad de panel} (ejemplo: granja) para permitir correlación en los $u_{it}$ dentro de un mismo $i$.
    
    \item En \textbf{STATA}:  
    \[
    \texttt{xtreg, fe i(PanelID) cluster(PanelID)}
    \]
    Esto produce inferencia válida siempre que el número de clusters sea suficientemente grande.
    
    \item Normalmente el interés está en $\beta$, pero los efectos $c_i$ también pueden ser relevantes:
    \begin{itemize}
        \item $\hat{c}_i$ del estimador de \textbf{variables dummy} es insesgado, pero no consistente para $c_i$ (con $T$ fijo y $N \to \infty$).
        \item \texttt{xtreg, fe} sustrae medias antes de estimar, por lo que no reporta $\hat{c}_i$.
        \item La constante reporta el promedio de los $\hat{c}_i$ entre unidades.
        \item Podemos recuperar $\hat{c}_i$ con:  
        \[
        \hat{c}_i = \bar{y}_i - \bar{x}_i \hat{\beta}
        \quad \; \; \; (\texttt{predict ci, u})
        \]
    \end{itemize}
\end{itemize}

\noindent\textbf{Intuición}
\begin{itemize}
    \item \textbf{Explicación:} La inferencia en panel requiere ajustar por correlación interna de los errores; además, los efectos $c_i$ pueden recuperarse pero no siempre son consistentes.
    \item \textbf{Ejemplo:} Al analizar productividad de granjas, los errores dentro de una misma granja están correlacionados; además, el efecto fijo de cada granja se puede calcular a partir de sus medias.
\end{itemize}

\subsection*{\noindent\textbf{Ejemplo: Democracia Directa y Naturalizaciones}}
\addcontentsline{toc}{subsection}{Ejemplo: Democracia Directa y Naturalizaciones}

\begin{itemize}
    \item Pregunta central: ¿\textbf{están en peor situación las minorías} bajo \textbf{democracia directa} que bajo \textbf{democracia representativa}?
    
    \item \textbf{Hainmueller y Hangartner (2012)} analizan datos de solicitudes de naturalización de inmigrantes en Suiza.
    
    \item En algunos municipios las solicitudes se votan en:
    \begin{itemize}
        \item \textbf{Referéndums} (\textbf{democracia directa}).
        \item \textbf{Consejos municipales electos} (\textbf{democracia representativa}).
    \end{itemize}
    
    \item Base de datos: panel anual de \textbf{1,400 municipios} en el periodo 1991–2009.
    
    \item Definición de variables:
    \[
    y_{it} = \text{tasa de naturalización} 
    = \frac{\text{número de naturalizaciones}_{it}}{\text{población extranjera elegible}_{it-1}}
    \]
    \[
    x_{it} =
    \begin{cases}
        1 & \text{si el municipio usa democracia representativa en } t \\
        0 & \text{si el municipio usa democracia directa en } t
    \end{cases}
    \]
\end{itemize}

\noindent\textbf{Intuición}
\begin{itemize}
    \item \textbf{Explicación:} El diseño permite comparar cómo cambia la tasa de naturalización dependiendo del mecanismo de decisión política en el municipio.
    \item \textbf{Ejemplo:} Dos municipios con características similares pueden diferir solo en el tipo de democracia, lo que permite identificar el efecto de la regla institucional.
\end{itemize}

\subsection*{\noindent\textbf{Datos de panel de naturalizaciones}}
\addcontentsline{toc}{subsection}{Datos de panel de naturalizaciones}

\begin{itemize}
    \item Base de datos en formato \textbf{panel} con información de municipios en Suiza.
    \item Variables principales:
\end{itemize}

\begin{table}[H]
\centering
\begin{tabular}{l l l}
\hline
\textbf{Variable} & \textbf{Tipo} & \textbf{Descripción} \\
\hline
\texttt{muniID}   & float  & Código del municipio \\
\texttt{muni\_name} & string & Nombre del municipio (1,430 municipios) \\
\texttt{year}     & int    & Año de observación \\
\texttt{nat\_rate} & float  & Tasa de naturalización (\%) \\
\texttt{repdem}   & float  & 1 = democracia representativa, 0 = democracia directa \\
\hline
\end{tabular}
\end{table}

\noindent\textbf{Intuición}
\begin{itemize}
    \item \textbf{Explicación:} El panel combina municipios y años, permitiendo analizar cómo la institución política ($repdem$) afecta la tasa de naturalización.
    \item \textbf{Ejemplo:} Si un municipio cambia de democracia directa a representativa, podemos observar cómo varía su tasa de naturalización en el tiempo.
\end{itemize}

\subsection*{\noindent\textbf{Datos de Panel. Long Format}}
\addcontentsline{toc}{subsection}{Datos de Panel. Long Format}

. \texttt{list muniID muni\_name year nat\_rate repdem in 31/40}

% --- Tabla de Datos con [H] ---
\begin{table}[H]
    \centering
    \begin{tabular}{|c|l|c|c|c|}
        \hline
        \textbf{muniID} & \textbf{muni\_name} & \textbf{year} & \textbf{nat\_rate} & \textbf{repdem} \\
        \hline
        2 & Affoltern A.A. & 2002 & 4.638365 & 0 \\
        2 & Affoltern A.A. & 2003 & 4.844814 & 0 \\
        2 & Affoltern A.A. & 2004 & 5.621302 & 0 \\
        2 & Affoltern A.A. & 2005 & 4.387827 & 0 \\
        2 & Affoltern A.A. & 2006 & 8.115358 & 1 \\
        \hline
        2 & Affoltern A.A. & 2007 & 7.067371 & 1 \\
        2 & Affoltern A.A. & 2008 & 8.977719 & 1 \\
        2 & Affoltern A.A. & 2009 & 6.119704 & 1 \\
        3 & Bonstetten & 1991 & 0.833334 & 0 \\
        3 & Bonstetten & 1992 & 0.8403362 & 0 \\
        \hline
    \end{tabular}
\end{table}

\vspace{0.5cm}

% --- Intuición: Sin subsección, solo texto en negrita y lista ---
\noindent\textbf{Intuición}
\begin{itemize}
    \item La clave del formato \textbf{Long} es que cada observación individual (\textbf{muniID-year}) tiene su propia \textbf{fila}, facilitando el seguimiento de los cambios temporales.
    \item El panel permite observar \textbf{variación en el tiempo} (diferentes años para el mismo municipio) y \textbf{variación entre unidades} (diferentes \textbf{muniID}).
\end{itemize}

\subsection*{\noindent\textbf{MCO Agrupados}}
\addcontentsline{toc}{subsection}{MCO Agrupados}

\begin{verbatim}
. reg nat_rate repdem , cl(muniID)
\end{verbatim}

\begin{center}
    \textbf{Regresión Lineal}
\end{center}

\begin{table}[H] % [H] requiere \usepackage{float}
    \centering
    \caption{Resultados de Regresión Lineal con Errores Agrupados}
    \resizebox{\textwidth}{!}{% Ajusta el tamaño de la tabla al ancho de texto si es necesario
    \begin{tabular}{lr @{\hskip 1cm} | @{\hskip 1cm} r}
        \multicolumn{2}{l}{Number of obs = 4655} \\
        \multicolumn{2}{l}{F( 1, 244) = 130.04} \\
        \multicolumn{2}{l}{Prob > F = 0.0000} \\
        \multicolumn{2}{l}{R-squared = 0.0748} \\
        \multicolumn{2}{l}{Root MSE = 3.98} \\
        \hline
        \multicolumn{5}{c}{(Std. Err. adjusted for 245 clusters in muniID)} \\
        \hline
        \textbf{nat\_rate} & \textbf{Coef.} & \textbf{Robust Std. Err.} & \textbf{t} & \textbf{P>|t|} & \textbf{[95\% Conf. Interval]} \\
        \hline
        repdem & 2.503318 & 0.2195202 & 11.40 & 0.000 & [2.070921 \ 2.935714] \\
        \_cons & 2.222683 & 0.10088 & 22.03 & 0.000 & [2.023976 \ 2.421389] \\
        \hline
    \end{tabular}
    } % Cierre de resizebox
\end{table}

\vspace{0.3cm}

% --- Intuición ---
\noindent\textbf{Intuición}
\begin{itemize}
    \item \textbf{Efecto Estimado:} El coeficiente de \textbf{repdem} (2.50) indica que la democracia representativa está asociada con un aumento de \textbf{2.5 puntos porcentuales} en la tasa de naturalización.
    \item \textbf{Agrupamiento (Clustering):} El uso de `cl(muniID)` corrige los errores estándar por \textbf{autocorrelación} dentro del municipio (correlación entre las observaciones a lo largo del tiempo para el mismo municipio), haciendo que la inferencia sea \textbf{válida} y \textbf{conservadora}.
\end{itemize}

\subsection*{\noindent\textbf{Descomposición de Varianza (within, between)}}
\addcontentsline{toc}{subsection}{Descomposición de Varianza (within, between)}

\begin{verbatim}
. tsset muniID year , yearly
panel variable: muniID (strongly balanced)
time variable: year, 1991 to 2009
delta: 1 year
\end{verbatim}

\begin{verbatim}
. xtsum nat_rate
\end{verbatim}

% --- Tabla de Resultados de xtsum ---
\begin{table}[H] % [H] requiere \usepackage{float}
    \centering
    \caption{Descomposición de Varianza de la Tasa de Naturalización (\texttt{nat\_rate})}
    \begin{tabular}{l | c c c c | r}
        \hline
        \textbf{Variable} & \textbf{Mean} & \textbf{Std. Dev.} & \textbf{Min} & \textbf{Max} & \textbf{Observations} \\
        \hline
        \textbf{nat\_rate} \_overall & 2.938992 & 4.137305 & 0 & 24.13793 & N = 4655 \\
        \hspace{5mm} between & & 1.622939 & 0 & 7.567746 & n = 245 \\
        \hspace{5mm} within & & 3.807039 & -3.711323 & 24.80134 & T = 19 \\
        \hline
    \end{tabular}
\end{table}

\vspace{0.3cm}

% --- Intuición ---
\noindent\textbf{Intuición}
\begin{itemize}
    \item \textbf{Varianza Between ($\sigma_{\mu}$):} Mide la variación \textbf{entre} municipios. Su desviación estándar (1.62) es relativamente \textbf{menor} que la *within*, sugiriendo que las diferencias \textbf{promedio} entre municipios son limitadas.
    \item \textbf{Varianza Within ($\sigma_{\epsilon}$):} Mide la variación \textbf{dentro} de cada municipio a lo largo del tiempo. Su desviación estándar (3.81) es \textbf{mayor} que la *between*, indicando que la mayor parte de la variación en \textbf{nat\_rate} proviene de los \textbf{cambios anuales} de la propia unidad.
\end{itemize}

\subsection*{\noindent\textbf{Within Estimator}}
\addcontentsline{toc}{subsection}{Within Estimator}

% Comandos Stata
\begin{verbatim}
. * get municipality means
. egen means_nat_rate = mean(nat_rate) , by(muniID)

. * compute deviations from means
. gen dm_nat_rate = nat_rate - means_nat_rate

. list muniID muni_name year nat_rate means_nat_rate dm_nat_rate in 20/40 , ab(20)
\end{verbatim}

% --- Tabla de Desviaciones de la Media ---
\begin{table}[H] % [H] requiere \usepackage{float}
    \centering
    \caption{Ilustración de la Transformación de Desviación de la Media}
    \begin{tabular}{|c|l|c|c|c|c|}
        \hline
        \textbf{muniID} & \textbf{muni\_name} & \textbf{year} & \textbf{nat\_rate} & \textbf{means\_nat\_rate} & \textbf{dm\_nat\_rate} \\
        \hline
        \hline
        2 & Affoltern A.A. & 1991 & 0.2173913 & 3.595932 & -3.37854 \\
        2 & Affoltern A.A. & 1992 & 0.9473684 & 3.595932 & -2.648563 \\
        2 & Affoltern A.A. & 1993 & 1.04712 & 3.595932 & -2.548811 \\
        2 & Affoltern A.A. & 1994 & 0.8342023 & 3.595932 & -2.761729 \\
        2 & Affoltern A.A. & 1995 & 2.002002 & 3.595932 & -1.59393 \\
        \hline
        2 & Affoltern A.A. & 1996 & 1.7769 & 3.595932 & -1.819031 \\
        2 & Affoltern A.A. & 1997 & 1.862745 & 3.595932 & -1.733186 \\
        2 & Affoltern A.A. & 1998 & 2.054155 & 3.595932 & -1.541776 \\
        2 & Affoltern A.A. & 1999 & 2.402135 & 3.595932 & -1.193796 \\
        \hline
    \end{tabular}
\end{table}

\vspace{0.3cm}

% --- Intuición ---
\noindent\textbf{Intuición}
\begin{itemize}
    \item \textbf{Eliminación del Sesgo:} La columna \textbf{dm\_nat\_rate} (\textbf{D}eviación de la \textbf{M}edia) representa la tasa de naturalización, pero eliminando la media de cada municipio (\textbf{means\_nat\_rate}).
    \item \textbf{Base de la Regresión:} Al restar la media de grupo, se eliminan los \textbf{efectos fijos} o características \textbf{invariantes en el tiempo} de cada municipio. La regresión del Estimador Within se ejecuta solo sobre estas variables **transformadas**.
\end{itemize}

\subsection*{\noindent\textbf{Efectos Fijos. Within Estimator}}
\addcontentsline{toc}{subsection}{Efectos Fijos. Within Estimator}

% Comandos Stata
\begin{verbatim}
. egen means_repdem = mean(repdem) , by(muniID)
. gen dm_repdem = repdem - means_repdem

. * regression with demeaned data
. reg dm_nat_rate dm_repdem , cl(muniID)
\end{verbatim}

\begin{center}
    \textbf{Regresión Lineal con Variables Transformadas}
\end{center}

\begin{table}[H] % [H] requiere \usepackage{float}
    \centering
    \caption{Resultados del Estimador Within (Efectos Fijos)}
    % Se ajusta la tabla para mayor claridad en las columnas numéricas
    \begin{tabular}{l c c c c c}
        \multicolumn{6}{l}{\textbf{Linear regression}} \\
        \hline
        \multicolumn{3}{l}{} & \multicolumn{3}{r}{Number of obs = 4655} \\
        \multicolumn{3}{l}{} & \multicolumn{3}{r}{F( 1, 244) = 265.18} \\
        \multicolumn{3}{l}{} & \multicolumn{3}{r}{Prob > F = 0.0000} \\
        \multicolumn{3}{l}{} & \multicolumn{3}{r}{R-squared = 0.1052} \\
        \multicolumn{3}{l}{} & \multicolumn{3}{r}{Root MSE = 3.6017} \\
        \hline
        \multicolumn{6}{c}{(Std. Err. adjusted for 245 clusters in muniID)} \\
        \hline
        \textbf{dm\_nat\_rate} & \textbf{Coef.} & \textbf{Robust Std. Err.} & \textbf{t} & \textbf{P>|t|} & \textbf{[95\% Conf. Interval]} \\
        \hline
        dm\_repdem & 3.0228 & 0.1856244 & 16.28 & 0.000 & [2.657169 \ 3.388431] \\
        \_cons & 6.65e-10 & 5.81e-09 & 0.11 & 0.909 & [-1.08e-08 \ 1.21e-08] \\
        \hline
    \end{tabular}
\end{table}

\vspace{0.3cm}

% --- Intuición ---
\noindent\textbf{Intuición}
\begin{itemize}
    \item \textbf{Interpretación del Coeficiente ($\mathbf{\text{dm\_repdem}}$):} El coeficiente de $\mathbf{3.02}$ mide el efecto causal puro de la \textbf{variación temporal} en la democracia (\textbf{repdem}) sobre la \textbf{variación temporal} en la tasa de naturalización, **eliminando** las características permanentes del municipio.
    \item \textbf{Comparación con MCO:} El efecto ($3.02$) es mayor que el MCO agrupado ($2.50$), sugiriendo que el MCO estaba sesgado a la baja (subestimado) el verdadero efecto, posiblemente debido a que factores \textbf{no observados} e invariantes estaban correlacionados con ambas variables.
\end{itemize}

\subsection*{\noindent\textbf{Efectos Fijos. Rutina programada: Stata}}
\addcontentsline{toc}{subsection}{Efectos Fijos. Rutina programada: Stata}

% Comando Stata
\begin{verbatim}
. xtreg nat_rate repdem , fe cl(muniID) i(muniID)
\end{verbatim}

\begin{table}[H] % [H] requiere \usepackage{float}
    \centering
    \caption{Resultados del Estimador de Efectos Fijos (xtreg, fe)}
    \begin{tabular}{l l r}
        \multicolumn{3}{l}{\textbf{Fixed-effects (within) regression}} \\
        \multicolumn{3}{l}{Group variable: muniID} \\
        \hline
        \multicolumn{3}{l}{R-sq: \hspace{5mm} within = 0.1052} \\
        \multicolumn{3}{l}{\hspace{17mm} between = 0.0005} \\
        \multicolumn{3}{l}{\hspace{17mm} overall = 0.0748} \\
        \multicolumn{3}{l}{corr(u\_i, Xb) = -0.1373} \\
    \end{tabular}
    % Segunda parte del encabezado
    \begin{tabular}{r @{\hskip 1cm} r}
        Number of obs = 4655 & F(1, 244) = 265.18 \\
        Number of groups = 245 & Prob > F = 0.0000 \\
        Obs per group: min = 19 & \\
        \hspace{16mm} avg = 19.0 & \\
        \hspace{16mm} max = 19 & \\
    \end{tabular}

    \vspace{0.2cm}
    \begin{tabular}{l c c c c c}
        \multicolumn{6}{c}{(Std. Err. adjusted for 245 clusters in muniID)} \\
        \hline
        \textbf{nat\_rate} & \textbf{Coef.} & \textbf{Robust Std. Err.} & \textbf{t} & \textbf{P>|t|} & \textbf{[95\% Conf. Interval]} \\
        \hline
        repdem & 3.0228 & 0.1856244 & 16.28 & 0.000 & [2.657169 \ 3.388431] \\
        \_cons & 2.074036 & 0.0531153 & 39.05 & 0.000 & [1.969413 \ 2.178659] \\
        \hline
        \multicolumn{6}{l}{} \\
        \multicolumn{6}{l}{\textbf{sigma\_u} = 1.7129711} \\
        \multicolumn{6}{l}{\textbf{sigma\_e} = 3.69998} \\
        \multicolumn{6}{l}{\textbf{rho} = 0.17650677 \hspace{2mm} (fraction of variance due to u\_i)} \\
        \hline
    \end{tabular}
\end{table}

\vspace{0.3cm}

% --- Intuición ---
\noindent\textbf{Intuición}
\begin{itemize}
    \item \textbf{Consistencia:} Los resultados de \textbf{xtreg, fe} confirman el coeficiente de **3.0228** encontrado con la regresión *demeaned*, validando la implementación de los \textbf{Efectos Fijos}.
    \item \textbf{Rho ($\rho$):} El valor de $\mathbf{\rho \approx 0.177}$ indica que aproximadamente el **17.7\%** de la varianza total del error se debe a los \textbf{efectos no observados invariantes} específicos de cada municipio ($\sigma_u$), lo que justifica el uso de un modelo de panel.
\end{itemize}

\subsection*{\noindent\textbf{Efectos Fijos. Rutina programada: Stata}}
\addcontentsline{toc}{subsection}{Efectos Fijos. Rutina programada: Stata}

% Comando Stata
\begin{verbatim}
. xtreg nat_rate repdem , fe cl(muniID) i(muniID)
\end{verbatim}

\begin{table}[H] % [H] requiere \usepackage{float}
    \centering
    \caption{Resultados del Estimador de Efectos Fijos (xtreg, fe)}
    \begin{tabular}{l l r}
        \multicolumn{3}{l}{\textbf{Fixed-effects (within) regression}} \\
        \multicolumn{3}{l}{Group variable: muniID} \\
        \hline
        \multicolumn{3}{l}{R-sq: \hspace{5mm} within = 0.1052} \\
        \multicolumn{3}{l}{\hspace{17mm} between = 0.0005} \\
        \multicolumn{3}{l}{\hspace{17mm} overall = 0.0748} \\
        \multicolumn{3}{l}{corr(u\_i, Xb) = -0.1373} \\
    \end{tabular}
    % Segunda parte del encabezado
    \begin{tabular}{r @{\hskip 1cm} r}
        Number of obs = 4655 & F(1, 244) = 265.18 \\
        Number of groups = 245 & Prob > F = 0.0000 \\
        Obs per group: min = 19 & \\
        \hspace{16mm} avg = 19.0 & \\
        \hspace{16mm} max = 19 & \\
    \end{tabular}

    \vspace{0.2cm}
    \begin{tabular}{l c c c c c}
        \multicolumn{6}{c}{(Std. Err. adjusted for 245 clusters in muniID)} \\
        \hline
        \textbf{nat\_rate} & \textbf{Coef.} & \textbf{Robust Std. Err.} & \textbf{t} & \textbf{P>|t|} & \textbf{[95\% Conf. Interval]} \\
        \hline
        repdem & 3.0228 & 0.1856244 & 16.28 & 0.000 & [2.657169 \ 3.388431] \\
        \_cons & 2.074036 & 0.0531153 & 39.05 & 0.000 & [1.969413 \ 2.178659] \\
        \hline
        \multicolumn{6}{l}{} \\
        \multicolumn{6}{l}{\textbf{sigma\_u} = 1.7129711} \\
        \multicolumn{6}{l}{\textbf{sigma\_e} = 3.69998} \\
        \multicolumn{6}{l}{\textbf{rho} = 0.17650677 \hspace{2mm} (fraction of variance due to u\_i)} \\
        \hline
    \end{tabular}
\end{table}

\vspace{0.3cm}

% --- Intuición ---
\noindent\textbf{Intuición}
\begin{itemize}
    \item \textbf{Consistencia:} Los resultados de \textbf{xtreg, fe} confirman el coeficiente de **3.0228** encontrado con la regresión *demeaned*, validando la implementación de los \textbf{Efectos Fijos}.
    \item \textbf{Rho ($\rho$):} El valor de $\mathbf{\rho \approx 0.177}$ indica que aproximadamente el **17.7\%** de la varianza total del error se debe a los \textbf{efectos no observados invariantes} específicos de cada municipio ($\sigma_u$), lo que justifica el uso de un modelo de panel.
\end{itemize}

\section*{\noindent\textbf{Aplicando efectos fijos}}
\addcontentsline{toc}{section}{Aplicando efectos fijos}

\begin{itemize}
    \item Los \textbf{efectos fijos} también pueden aplicarse en otras estructuras de datos, permitiendo restringir las comparaciones dentro de la misma unidad.
    
    \item \textbf{Ejemplos:}
    \begin{itemize}
        \item \textbf{Parejas apareadas:} usar efectos fijos de gemelos para controlar características familiares no observadas.
        \item \textbf{Escuelas:} incluir efectos fijos de escuela para capturar factores no observados propios de cada institución.
    \end{itemize}
\end{itemize}

\noindent\textbf{Intuición}
\begin{itemize}
    \item \textbf{Explicación:} El método de efectos fijos funciona como un “control interno” que elimina factores constantes dentro de cada unidad.
    \item \textbf{Ejemplo:} Comparar gemelos elimina el efecto de los antecedentes familiares; comparar alumnos dentro de la misma escuela elimina el efecto del contexto escolar fijo.
\end{itemize}

\section*{\noindent\textbf{Problemas que los efectos fijos no resuelven}}
\addcontentsline{toc}{section}{Problemas que los efectos fijos no resuelven}

\begin{itemize}
    \item Modelo de referencia:
    \[
    y_{it} = x_{it}\beta + c_i + u_{it}, \quad t = 1,2,\ldots,T
    \]
    donde:
    \[
    y_{it} = \text{tasa de homicidios}, \quad x_{it} = \text{gasto en policía per cápita}
    \]
    
    \item Si corremos una regresión de $y$ sobre $x$ con efectos fijos por ciudad, el estimador $\hat{\beta}_{FE}$ será \textbf{inconsistente} salvo que se cumpla la \textbf{estricta exogeneidad}:
    \[
    E(u_{it} \mid x_{i1}, x_{i2}, \ldots, x_{iT}, c_i) = 0, \quad t = 1,2,\ldots,T
    \]
    
    \item Esto implica que $u_{it}$ no debe estar correlacionado con regresores pasados, actuales ni futuros.
    
    \item \textbf{Violaciones comunes:}
    \begin{enumerate}
        \item \textbf{Variables omitidas que varían en el tiempo:} un boom económico puede llevar a más gasto policial y, al mismo tiempo, a menos homicidios.
        \item \textbf{Simultaneidad:} si la ciudad ajusta el gasto policial en respuesta a tasas de homicidios pasadas, entonces $x_{t+1}$ estará correlacionado con $u_t$.  
        Una $x$ estrictamente exógena no puede reaccionar a lo que le ocurre a $y$ en el pasado o el futuro.
    \end{enumerate}
\end{itemize}

\noindent\textbf{Intuición}
\begin{itemize}
    \item \textbf{Explicación:} Los efectos fijos eliminan solo factores constantes; no corrigen problemas de variables omitidas dinámicas ni de causalidad inversa.
    \item \textbf{Ejemplo:} Si una ciudad aumenta la policía porque hubo más homicidios el año pasado, el gasto y los errores pasados estarán correlacionados, rompiendo la exogeneidad.
\end{itemize}

\section*{\noindent\textbf{Efectos Aleatorios}}
\addcontentsline{toc}{section}{Efectos Aleatorios}

\begin{itemize}
    \item Consideremos nuevamente el modelo:
    \[
    y_{it} = x_{it}\beta + c_i + u_{it}, \quad t = 1,2,\ldots,T
    \]
    
    \item Un problema del modelo de \textbf{efectos fijos} es que no puede estimar el impacto de \textbf{regresores invariantes en el tiempo} (ejemplo: calidad del suelo, ubicación de la granja).
    
    \item Como el estimador FE permite correlación entre $c_i$ y $x_{it}$, no es posible separar el efecto de esos regresores constantes del efecto fijo $c_i$.
    
    \item Para estimarlos, necesitamos el supuesto de \textbf{ortogonalidad}:  
    \[
    \text{Cov}(x_{it}, c_i) = 0, \quad \forall t=1,\ldots,T
    \]
    
    \item Este es un \textbf{supuesto fuerte}: se requiere que los efectos no observados $c_i$ no estén correlacionados con ninguna variable explicativa en $x_{it}$ en ningún periodo.
    
    \item Ejemplo: si incluimos \textbf{calidad del suelo} en $x_{it}$, debemos suponer que no se correlaciona con otros insumos invariantes en el tiempo (como ubicación geográfica).
\end{itemize}

\noindent\textbf{Intuición}
\begin{itemize}
    \item \textbf{Explicación:} El modelo de efectos aleatorios permite identificar variables constantes en el tiempo, pero exige independencia fuerte entre $c_i$ y los regresores.
    \item \textbf{Ejemplo:} Es como analizar granjas con diferentes calidades de suelo: para que el modelo funcione, debemos asumir que la calidad del suelo no está relacionada con otros factores fijos como el clima de la región.
\end{itemize}

\section*{\noindent\textbf{Métodos de efectos aleatorios (RE)}}
\addcontentsline{toc}{section}{Métodos de efectos aleatorios (RE)}

\begin{itemize}
    \item Los modelos de \textbf{efectos aleatorios} incluyen $c_i$ dentro del término de error, bajo el supuesto de que $c_i$ es \textbf{ortogonal} a $x_{it}$, y corrigen la \textbf{correlación serial} en el error compuesto.
    
    \item Estos modelos imponen dos condiciones: \textbf{exogeneidad estricta} y \textbf{ortogonalidad} entre $c_i$ y $x_{it}$.
    
    \item Supuestos (RE1):
    \[
    E(u_{it} \mid x_i, c_i) = 0, \quad t=1,2,\ldots,T
    \]
    \[
    E(c_i \mid x_i) = E(c_i) = 0, \quad \text{con } x_i = (x_{i1}, x_{i2}, \ldots, x_{iT})
    \]
    
    \item Lo importante es que:
    \[
    E(c_i \mid x_i) = E(c_i)
    \]
    El supuesto $E(c_i)=0$ no implica pérdida de generalidad siempre que haya una constante incluida en $x_{it}$.
    
    \item Bajo este supuesto, incluso \textbf{MCO} sería \textbf{consistente}, pero no \textbf{eficiente}.  
    Por eso, se emplea \textbf{MCG (GLS)} para obtener eficiencia.
\end{itemize}

\noindent\textbf{Intuición}
\begin{itemize}
    \item \textbf{Explicación:} El modelo RE trata los efectos individuales como parte del error, suponiendo que no están correlacionados con los regresores, y usa GLS para mejorar la eficiencia.
    \item \textbf{Ejemplo:} Es como analizar escuelas donde la “calidad promedio” de cada escuela se asume independiente del gasto en insumos: si se cumple, se puede estimar con más precisión que con FE.
\end{itemize}

\hrule

\begin{itemize}
    \item El enfoque de \textbf{efectos aleatorios} considera explícitamente la \textbf{correlación serial} en el error compuesto:
    \[
    v_{it} = c_i + u_{it}
    \]
    
    \item El modelo de regresión puede escribirse como:
    \[
    y_{it} = x_{it}\beta + v_{it}
    \]
    
    \item Bajo los supuestos de efectos aleatorios:
    \[
    E(v_{it} \mid x_i) = 0, \quad t=1,2,\ldots,T
    \]
    
    \item En consecuencia, podemos aplicar \textbf{MCG (GLS)} para tener en cuenta la estructura particular del error compuesto $v_{it} = c_i + u_{it}$.
\end{itemize}

\noindent\textbf{Intuición}
\begin{itemize}
    \item \textbf{Explicación:} El error compuesto $v_{it}$ tiene una parte común ($c_i$) y otra idiosincrática ($u_{it}$), lo que genera correlación entre periodos para un mismo individuo.
    \item \textbf{Ejemplo:} Es como medir la productividad de una persona: siempre habrá un componente fijo de talento ($c_i$) más choques aleatorios cada año ($u_{it}$); GLS ajusta por esa estructura.
\end{itemize}

\hrule

\begin{itemize}
    \item El modelo puede escribirse para todos los periodos como:
    \[
    y_i = X_i \beta + v_i
    \]
    donde $y_i$ es un vector $T \times 1$, $X_i$ es una matriz $T \times K$, y $v_i$ el vector de errores compuestos.
    
    \item Definimos la matriz de varianza no condicional de $v_i$ como:
    \[
    \Omega \equiv E(v_i v_i^T)
    \]
    
    \item $\Omega$ es una matriz de dimensión $T \times T$, y es \textbf{la misma para todos los individuos $i$} porque se asume \textbf{muestreo aleatorio} en la dimensión transversal.
    
    \item Para la consistencia del estimador GLS necesitamos la condición de rango habitual:
    \[
    \text{RE2: } \quad \text{rank}\!\left( E(X_i^T \Omega^{-1} X_i) \right) = K
    \]
\end{itemize}

\noindent\textbf{Intuición}
\begin{itemize}
    \item \textbf{Explicación:} La estructura del error compuesto implica una matriz de covarianza común para todos los individuos; GLS usa esta matriz para ponderar correctamente las observaciones.
    \item \textbf{Ejemplo:} Es como dar más o menos peso a observaciones de un mismo individuo dependiendo de cuánta correlación exista entre sus periodos.
\end{itemize}

\hrule


\begin{itemize}
    \item Un análisis estándar de efectos aleatorios añade supuestos sobre los \textbf{errores idiosincráticos}, lo que da a $\Omega$ una forma particular.
    
    \item Supuestos (RE3):
    \[
    E(u_i u_i^T \mid x_i, c_i) = \sigma_u^2 I_T
    \]
    \[
    E(c_i^2 \mid x_i) = \sigma_c^2
    \]
    
    \item Bajo estos supuestos, la matriz de varianza $\Omega$ es:
    \[
    \Omega =
    \begin{bmatrix}
    \sigma_c^2 + \sigma_u^2 & \sigma_c^2 & \cdots & \sigma_c^2 \\
    \sigma_c^2 & \sigma_c^2 + \sigma_u^2 & \cdots & \sigma_c^2 \\
    \vdots & \vdots & \ddots & \vdots \\
    \sigma_c^2 & \sigma_c^2 & \cdots & \sigma_c^2 + \sigma_u^2
    \end{bmatrix}
    \]
    
    \item Es decir, la matriz $\Omega$ tiene $\sigma_c^2 + \sigma_u^2$ en la diagonal, y $\sigma_c^2$ en todas las posiciones fuera de la diagonal.
\end{itemize}

\noindent\textbf{Intuición}
\begin{itemize}
    \item \textbf{Explicación:} La parte fija $c_i$ genera correlación perfecta entre todos los periodos de un mismo individuo, mientras que $u_{it}$ añade variación idiosincrática independiente.
    \item \textbf{Ejemplo:} Para un trabajador, $c_i$ puede reflejar su productividad innata (común a todos los años), mientras que $u_{it}$ recoge shocks de productividad que cambian cada año.
\end{itemize}

\hrule

\begin{itemize}
    \item Si contamos con estimadores consistentes de $\sigma_c^2$ y $\sigma_u^2$  
    (ver \textbf{Wooldridge, p. 260} sobre cómo obtenerlos), podemos construir un estimador consistente de la matriz de varianza:
    \[
    \hat{\Omega} \equiv \hat{\sigma}_u^2 I_T + \hat{\sigma}_c^2 j_T j_T^T
    \]
    donde $j_T j_T^T$ es una matriz $T \times T$ con unos en todas sus entradas.
    
    \item Con esta matriz, el estimador de \textbf{efectos aleatorios} es:
    \[
    \hat{\beta}_{RE} 
    = \left( \sum_{i=1}^N X_i^T \hat{\Omega}^{-1} X_i \right)^{-1} 
      \left( \sum_{i=1}^N X_i^T \hat{\Omega}^{-1} y_i \right)
    \]
    
    \item El estimador RE es un caso particular de \textbf{GLS factible}, donde solo se deben estimar dos parámetros en la matriz de varianza-covarianza.
    
    \item Si los supuestos de RE se cumplen, este estimador es \textbf{consistente} y \textbf{eficiente}.
\end{itemize}

\noindent\textbf{Intuición}
\begin{itemize}
    \item \textbf{Explicación:} El método RE estima la estructura de la varianza entre y dentro de individuos y aplica GLS factible para ponderar las observaciones.
    \item \textbf{Ejemplo:} Es como ajustar la importancia relativa de variaciones entre municipios y dentro de cada municipio al estimar el efecto de una política.
\end{itemize}

\section*{\noindent\textbf{Regresión de efectos aleatorios (RE)}}
\addcontentsline{toc}{section}{Regresión de efectos aleatorios (RE)}

% Comando Stata
\begin{verbatim}
. xtreg nat_rate repdem , re cl(muniID) i(muniID)
\end{verbatim}

% --- Tabla de Resultados de xtreg, re ---
\begin{table}[H] % [H] requiere \usepackage{float}
    \centering
    \caption{Resultados del Estimador de Efectos Aleatorios (xtreg, re)}
    \begin{tabular}{l l r}
        \multicolumn{3}{l}{\textbf{Random-effects GLS regression}} \\
        \multicolumn{3}{l}{Group variable: muniID} \\
        \hline
        \multicolumn{3}{l}{R-sq: \hspace{5mm} within = 0.1052} \\
        \multicolumn{3}{l}{\hspace{17mm} between = 0.0005} \\
        \multicolumn{3}{l}{\hspace{17mm} overall = 0.0748} \\
        \multicolumn{3}{l}{corr(u\_i, X) = 0 (assumed)} \\
    \end{tabular}
    % Segunda parte del encabezado
    \begin{tabular}{r @{\hskip 1cm} r}
        Number of obs = 4655 & Wald chi2(1) = 227.99 \\
        Number of groups = 245 & Prob > chi2 = 0.0000 \\
        Obs per group: min = 19 & \\
        \hspace{16mm} avg = 19.0 & \\
        \hspace{16mm} max = 19 & \\
    \end{tabular}

    \vspace{0.2cm}
    \begin{tabular}{l c c c c c}
        \multicolumn{6}{c}{(Std. Err. adjusted for 245 clusters in muniID)} \\
        \hline
        \textbf{nat\_rate} & \textbf{Coef.} & \textbf{Robust Std. Err.} & \textbf{z} & \textbf{P>|z|} & \textbf{[95\% Conf. Interval]} \\
        \hline
        repdem & 2.859397 & 0.1893742 & 15.10 & 0.000 & [2.48823 \ 3.230564] \\
        \_cons & 2.120793 & 0.0972959 & 21.80 & 0.000 & [1.930096 \ 2.311489] \\
        \hline
        \multicolumn{6}{l}{\textbf{sigma\_u} = 1.3866768} \\
        \multicolumn{6}{l}{\textbf{sigma\_e} = 3.69998} \\
        \multicolumn{6}{l}{\textbf{rho} = 0.1231606 \hspace{2mm} (fraction of variance due to u\_i)} \\
        \hline
    \end{tabular}
\end{table}

\vspace{0.3cm}

% --- Intuición ---
\noindent\textbf{Intuición}
\begin{itemize}
    \item \textbf{Supuesto Clave:} El estimador de \textbf{Efectos Aleatorios (RE)} asume que los efectos no observados ($\mathbf{u_i}$) **no están correlacionados** con la variable explicativa ($\mathbf{repdem}$). Esta es la principal diferencia con el modelo FE.
    \item \textbf{Eficiencia vs. Sesgo:} El coeficiente RE ($\mathbf{2.86}$) está entre el MCO ($2.50$) y el FE ($3.02$). RE es más **eficiente** (usa variación entre y dentro), pero si el supuesto de no correlación es falso (como sugiere la prueba de Hausman), sus estimaciones serán **sesgadas**.
\end{itemize}

\section*{\noindent\textbf{RE versus FE, FD o Variables Dummy}}
\addcontentsline{toc}{section}{RE versus FE, FD o Variables Dummy}

\begin{itemize}
    \item El supuesto central de \textbf{RE}, $E(c_i \mid x_i) = E(c_i) = 0$, es muy \textbf{fuerte} y poco realista en muchos contextos.
    
    \item Por ello, en la práctica suelen preferirse los \textbf{estimadores de efectos fijos}.
    
    \item Pregunta clave: ¿\textbf{qué estimador de efectos fijos elegir}?
    
    \item Con solo \textbf{dos periodos}, los estimadores \textbf{FE}, \textbf{FD} y de \textbf{variables dummy} son idénticos.
    
    \item Si $T > 2$, la elección depende de los \textbf{supuestos sobre los errores} $u_{it}$:
    \begin{itemize}
        \item En la práctica, se suele asumir \textbf{FE3} (no autocorrelación serial) y usar el estimador \textbf{within (FE)}.
        \item Si el interés está en recuperar los $c_i$, puede emplearse el estimador de \textbf{variables dummy}.
    \end{itemize}
\end{itemize}

\noindent\textbf{Intuición}
\begin{itemize}
    \item \textbf{Explicación:} Los métodos FE, FD y dummies son robustos a la correlación entre $c_i$ y $x_{it}$; RE solo funciona bajo el supuesto más restrictivo de independencia.
    \item \textbf{Ejemplo:} Con pocos años de datos, todas las técnicas dan lo mismo; con más periodos, la elección depende de cómo supongamos que se comportan los errores.
\end{itemize}

\section*{\noindent\textbf{Prueba de Hausman}}
\addcontentsline{toc}{section}{Prueba de Hausman}

\begin{itemize}
    \item Comparación entre estimadores \textbf{RE} y \textbf{FE}:
\end{itemize}

\begin{table}[H]
\centering
\begin{tabular}{lcc}
\hline
 & $\hat{\beta}_{RE}$ & $\hat{\beta}_{FE}$ \\
\hline
$H_0: \; \text{Cov}[x_{it}, c_i] = 0$ & Consistente y eficiente & Consistente \\
$H_1: \; \text{Cov}[x_{it}, c_i] \neq 0$ & Inconsistente & Consistente \\
\hline
\end{tabular}
\end{table}

\begin{itemize}
    \item Bajo $H_0$, la diferencia $\hat{\beta}_{RE} - \hat{\beta}_{FE}$ debe ser cercana a cero.
    \item Bajo $H_1$, esta diferencia será significativa.
    
    \item El estadístico de prueba (para muestras grandes) es:
    \[
    (\hat{\beta}_{FE} - \hat{\beta}_{RE})^T 
    \left( \widehat{\text{Var}}[\hat{\beta}_{FE}] - \widehat{\text{Var}}[\hat{\beta}_{RE}] \right)^{-1} 
    (\hat{\beta}_{FE} - \hat{\beta}_{RE})
    \;\; \overset{d}{\longrightarrow} \; \chi^2_k
    \]
    donde $k$ es el número de regresores que varían en el tiempo.
    
    \item Si se rechaza $H_0$, concluimos que los efectos aleatorios no son apropiados y preferimos la especificación \textbf{FE}.
\end{itemize}

\noindent\textbf{Intuición}
\begin{itemize}
    \item \textbf{Explicación:} La prueba de Hausman compara directamente los estimadores RE y FE; si difieren mucho, RE es inconsistente.
    \item \textbf{Ejemplo:} Es como contrastar dos termómetros: si uno siempre mide más alto que otro, asumimos que está mal calibrado (RE) y confiamos en el más robusto (FE).
\end{itemize}

\subsection*{\noindent\textbf{Regresión de efectos aleatorios (RE)}}
\addcontentsline{toc}{subsection}{Regresión de efectos aleatorios (RE)}

% Comandos Stata
\begin{verbatim}
. quietly: xtreg nat_rate repdem , fe i(muniID)
. estimates store FE

. quietly: xtreg nat_rate repdem , re i(muniID)
. estimates store RE

. hausman FE RE
\end{verbatim}

% --- Tabla de Resultados de la Prueba de Hausman ---
\begin{table}[H] % [H] requiere \usepackage{float}
    \centering
    \caption{Prueba de Hausman para la Elección del Modelo de Panel}
    \begin{tabular}{l c c c c}
        \multicolumn{5}{c}{\hrulefill \textbf{Coefficients} \hrulefill} \\
        & (b) & (B) & (b-B) & sqrt(diag(V\_b-V\_B)) \\
        & \textbf{FE} & \textbf{RE} & \textbf{Difference} & \textbf{S.E.} \\
        \hline
        repdem & 3.0228 & 2.859397 & 0.1634027 & 0.0304517 \\
        \hline
    \end{tabular}

    \vspace{0.2cm}
    \begin{itemize}
        \item[] b = consistent under Ho and Ha; obtained from xtreg
        \item[] B = inconsistent under Ha, efficient under Ho; obtained from xtreg
    \end{itemize}

    \vspace{0.2cm}
    \noindent\textbf{Test:} Ho: difference in coefficients not systematic

    \begin{verbatim}
    chi2(1) = (b-B)'[(V_b-V_B)^(-1)](b-B)
            = 28.79
    \end{verbatim}
\end{table}

\vspace{0.3cm}

% --- Intuición ---
\noindent\textbf{Intuición}
\begin{itemize}
    \item \textbf{Hipótesis Nula ($\mathbf{H_0}$):} La diferencia entre los coeficientes $\mathbf{FE}$ y $\mathbf{RE}$ \textbf{no es sistemática} (RE es consistente y eficiente).
    \item \textbf{Conclusión:} Dado que el estadístico $\mathbf{\chi^2(1) = 28.79}$ es grande y el valor $\mathbf{p}$ asociado es muy pequeño (no visible, pero el valor es significativo), se \textbf{rechaza la hipótesis nula ($\mathbf{H_0}$)}. Por lo tanto, el modelo de \textbf{Efectos Fijos (FE)} es el estimador \textbf{consistente} preferido.
\end{itemize}

\section*{\noindent\textbf{Prueba de Hausman}}
\addcontentsline{toc}{section}{Prueba de Hausman}

\begin{itemize}
    \item La \textbf{prueba de Hausman} no evalúa si el modelo de efectos fijos es correcto; lo que asume es que el estimador de efectos fijos es \textbf{consistente}.
    
    \item La versión convencional de la prueba usa un modelo de \textbf{errores homocedásticos} y no permite el uso de \textbf{clustering}.
    
    \item Existen variantes de la prueba (como la \textbf{Sargan-Hansen}) que permiten aplicar \textbf{clustering} a los errores estándar.
    
    \item Ejemplo en \textbf{STATA}:
    \begin{verbatim}
    * hausman test with clustering
    quietly: xtreg nat_rate repdem , re i(muniID) cl(muniID)
    xtoverid
    \end{verbatim}
    
    \item Resultado de prueba (ejemplo):  
    \[
    \text{Sargan-Hansen statistic} = 26.560, \quad \chi^2(1), \quad p\text{-value} = 0.0000
    \]
    lo que lleva a rechazar la hipótesis nula y preferir la especificación de efectos fijos.
\end{itemize}

\noindent\textbf{Intuición}
\begin{itemize}
    \item \textbf{Explicación:} La Hausman clásica compara RE vs FE, pero para datos de panel con correlación interna se recomienda una versión robusta al clustering.
    \item \textbf{Ejemplo:} En los municipios suizos, si el p-valor es muy bajo, concluimos que los efectos aleatorios no son válidos y que debemos usar efectos fijos.
\end{itemize}

\section*{\noindent\textbf{Añadiendo efectos fijos en el tiempo}}
\addcontentsline{toc}{section}{Añadiendo efectos fijos en el tiempo}

\begin{itemize}
    \item Modelo de referencia:
    \[
    y_{it} = x_{it}\beta + c_i + u_{it}, \quad t = 1,2,\ldots,T
    \]
    
    \item Supuesto de efectos fijos:
    \[
    E(u_{it} \mid x_{i1}, x_{i2}, \ldots, x_{iT}, c_i) = 0, 
    \quad t=1,2,\ldots,T
    \]
    donde $u_{it}$ no se correlaciona con regresores pasados, presentes ni futuros.  
    Es decir, los regresores son \textbf{estrictamente exógenos} condicionados a $c_i$.
    
    \item \textbf{Violación típica:} choques comunes que afectan a todas las unidades de manera similar y están correlacionados con $x_{it}$.
    
    \item Ejemplos de choques comunes:
    \begin{itemize}
        \item \textbf{Tendencias tecnológicas o climáticas} que influyen en la productividad agrícola.
        \item \textbf{Tendencias migratorias} que impactan las tasas de naturalización.
    \end{itemize}
    
    \item Para capturar estos efectos, podemos incluir \textbf{efectos fijos de tiempo} en el modelo.
\end{itemize}

\noindent\textbf{Intuición}
\begin{itemize}
    \item \textbf{Explicación:} Los efectos fijos de tiempo permiten controlar choques comunes a todas las unidades en un periodo, evitando que sesguen la estimación.
    \item \textbf{Ejemplo:} Si en un año hay una reforma migratoria que afecta a todos los municipios, el efecto fijo de tiempo controla ese cambio común y evita confundirlo con el impacto de $x_{it}$.
\end{itemize}

\hrule

\begin{itemize}
    \item \textbf{Tendencia lineal común:}
    \[
    y_{it} = x_{it}\beta + c_i + t + u_{it}, \quad t=1,2,\ldots,T
    \]
    Una tendencia de tiempo lineal que aplica a todas las unidades.

    \item \textbf{Efectos fijos de tiempo:}
    \[
    y_{it} = x_{it}\beta + c_i + \tau_t + u_{it}, \quad t=1,2,\ldots,T
    \]
    Captura choques comunes a todas las unidades en cada periodo.  
    Este modelo puede interpretarse como una \textbf{generalización de diferencias en diferencias}.
    
    \item \textbf{Tendencias específicas por unidad:}
    \[
    y_{it} = x_{it}\beta + c_i + g_i \times t + \tau_t + u_{it}, \quad t=1,2,\ldots,T
    \]
    Permite que cada unidad $i$ tenga su propia tendencia lineal en el tiempo.
\end{itemize}

\noindent\textbf{Intuición}
\begin{itemize}
    \item \textbf{Explicación:} Los efectos fijos de tiempo y las tendencias permiten capturar dinámicas comunes o específicas que evolucionan con el tiempo y podrían sesgar la estimación.
    \item \textbf{Ejemplo:} Una política nacional que afecta a todos los municipios se recoge con $\tau_t$, mientras que un municipio en crecimiento constante en su productividad agrícola se modela con una tendencia $g_i \times t$.
\end{itemize}

\section*{\noindent\textbf{Modelando los efectos en el tiempo}}
\addcontentsline{toc}{section}{Modelando los efectos en el tiempo}

\begin{figure}[H]
    \centering
    \includegraphics[width=0.7\textwidth]{image1}
    \caption{\footnotesize Representación gráfica de tendencias temporales: 
    (1) tendencia lineal común, 
    (2) efectos fijos de tiempo, 
    (3) tendencias lineales específicas por unidad.}
\end{figure}

\noindent\textbf{Intuición}
\begin{itemize}
    \item \textbf{Explicación:} Una tendencia común en el tiempo afecta a todas las unidades igual; los efectos fijos de tiempo permiten choques puntuales en cada periodo; las tendencias específicas dejan que cada unidad evolucione de forma diferente.
    \item \textbf{Ejemplo:} Todos los países pueden enfrentar un shock global en un año (efecto fijo de tiempo), pero algunos países pueden mostrar trayectorias crecientes y otros decrecientes en productividad (tendencias específicas).
\end{itemize}

\section*{\noindent\textbf{Efectos fijos: añadiendo efectos en el tiempo}}
\addcontentsline{toc}{section}{Efectos fijos: añadiendo efectos en el tiempo}

\begin{itemize}
    \item Para incluir \textbf{efectos fijos de tiempo}, podemos crear una variable que agrupe los años de la muestra.
    
    \item En \textbf{STATA}:
    \begin{verbatim}
    . egen time = group(year)
    . list muniID muni_name year time in 20/40 , ab(20)
    \end{verbatim}
    
    \item Ejemplo de salida:
\end{itemize}

\begin{table}[H]
\centering
\begin{tabular}{cccc}
\hline
\textbf{muniID} & \textbf{muni\_name} & \textbf{year} & \textbf{time} \\
\hline
2 & Affoltern A.A. & 1991 & 1 \\
2 & Affoltern A.A. & 1992 & 2 \\
2 & Affoltern A.A. & 1993 & 3 \\
2 & Affoltern A.A. & 1994 & 4 \\
2 & Affoltern A.A. & 1995 & 5 \\
2 & Affoltern A.A. & 1996 & 6 \\
2 & Affoltern A.A. & 1997 & 7 \\
2 & Affoltern A.A. & 1998 & 8 \\
2 & Affoltern A.A. & 1999 & 9 \\
2 & Affoltern A.A. & 2000 & 10 \\
2 & Affoltern A.A. & 2001 & 11 \\
\hline
\end{tabular}
\end{table}

\noindent\textbf{Intuición}
\begin{itemize}
    \item \textbf{Explicación:} Al agrupar los años en una variable $time$, podemos incluir \textbf{dummies por periodo}, capturando choques comunes en cada año.
    \item \textbf{Ejemplo:} Si en 1995 hubo una crisis que afectó a todos los municipios, un efecto fijo de tiempo recoge ese impacto común.
\end{itemize}

\section*{\noindent\textbf{Efectos fijos: tendencia lineal en el tiempo}}
\addcontentsline{toc}{section}{Efectos fijos: tendencia lineal en el tiempo}

\begin{itemize}
    \item Ejemplo en \textbf{STATA} con tendencia lineal:
    \begin{verbatim}
    . xtreg nat_rate repdem time , fe cl(muniID) i(muniID)
    \end{verbatim}
    
    \item Resultados principales:
    \begin{itemize}
        \item $R^2_{within} = 0.1604$, $R^2_{between} = 0.0005$, $R^2_{overall} = 0.1350$.
        \item Test global: $F(2,244) = 247.57$, p-valor = 0.0000.
        \item Coeficientes estimados:
        \[
        \begin{aligned}
        \hat{\beta}_{repdem} &= 0.8248 \quad (p = 0.002) \\
        \hat{\beta}_{time} &= 0.2314 \quad (p = 0.000) \\
        \text{constante} &= 0.3893 \quad (p = 0.003)
        \end{aligned}
        \]
        \item Estimaciones de varianzas:
        \[
        \sigma_u = 1.627, \quad \sigma_e = 3.584, \quad \rho = 0.171
        \]
        donde $\rho$ mide la proporción de varianza atribuida al componente $u_i$.
    \end{itemize}
\end{itemize}

\noindent\textbf{Intuición}
\begin{itemize}
    \item \textbf{Explicación:} Incluir una tendencia lineal en el tiempo permite capturar una evolución sistemática en la tasa de naturalización, además de los efectos fijos por municipio.
    \item \textbf{Ejemplo:} Si existe un aumento gradual en la naturalización de inmigrantes en todos los municipios a lo largo del tiempo, el término $time$ lo recoge y evita sesgo en la estimación de $repdem$.
\end{itemize}

\section*{\noindent\textbf{Efectos fijos: efectos fijos por año}}
\addcontentsline{toc}{section}{Efectos fijos: efectos fijos por año}

\begin{itemize}
    \item Ejemplo en \textbf{STATA} con efectos fijos de tiempo:
    \begin{verbatim}
    . xtreg nat_rate repdem i.time , fe cl(muniID) i(muniID)
    \end{verbatim}
    
    \item Resultados principales:
    \begin{itemize}
        \item $R^2_{within} = 0.1885$, $R^2_{between} = 0.0005$, $R^2_{overall} = 0.1575$.
        \item Test global: $F(19,244) = 31.48$, p-valor = 0.0000.
        \item Coeficientes estimados:
        \[
        \hat{\beta}_{repdem} = 1.2037 \quad (p = 0.000)
        \]
        \item Ejemplos de efectos fijos de tiempo:
        \[
        \text{time2} = 0.3829 \quad (p = 0.027), \quad
        \text{time3} = 0.2789 \quad (p = 0.188), \quad
        \text{time4} = 0.7034 \quad (p = 0.000)
        \]
    \end{itemize}
    
    \item Estos coeficientes capturan \textbf{choques específicos por año}, comunes a todos los municipios.
\end{itemize}

\noindent\textbf{Intuición}
\begin{itemize}
    \item \textbf{Explicación:} Incluir dummies de año permite controlar factores comunes en cada periodo (reformas, crisis, cambios políticos) que podrían sesgar las estimaciones.
    \item \textbf{Ejemplo:} Si en 1994 hubo una reforma migratoria que afectó a todos los municipios, el efecto fijo de ese año lo captura y evita atribuirlo erróneamente a $repdem$.
\end{itemize}

\section*{\noindent\textbf{Efectos fijos: efectos de tiempo por unidad}}
\addcontentsline{toc}{section}{Efectos fijos: efectos de tiempo por unidad}

\begin{itemize}
    \item Ejemplo en \textbf{STATA} con interacción unidad × tiempo:
    \begin{verbatim}
    . xtreg nat_rate repdem muniID#c.time i.time , fe cl(muniID) i(muniID)
    \end{verbatim}
    
    \item Este modelo permite que cada unidad (municipio) tenga su \textbf{propia tendencia temporal}.
    
    \item Resultados principales:
    \begin{itemize}
        \item $R^2_{within} = 0.2650$, $R^2_{between} = 0.5185$, $R^2_{overall} = 0.2864$.
        \item Coeficiente de interés:
        \[
        \hat{\beta}_{repdem} = 0.9865 \quad (p = 0.002)
        \]
        \item Ejemplos de pendientes unitarias estimadas (interacciones municipio × tiempo):
        \[
        1: 0.3334 \quad (p = 0.000), \quad
        2: 0.2914 \quad (p = 0.000), \quad
        3: 0.2499 \quad (p = 0.000)
        \]
    \end{itemize}
    
    \item Nota: la categoría de referencia (año 19) fue omitida por colinealidad.
\end{itemize}

\noindent\textbf{Intuición}
\begin{itemize}
    \item \textbf{Explicación:} Este modelo es una versión extendida de efectos fijos que permite que cada unidad tenga su propia tendencia temporal, capturando evoluciones heterogéneas.
    \item \textbf{Ejemplo:} Mientras un municipio puede mostrar una tendencia creciente en naturalizaciones, otro puede permanecer estable o incluso decrecer; las interacciones capturan estas diferencias.
\end{itemize}





\end{document}
